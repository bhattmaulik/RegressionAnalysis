% Options for packages loaded elsewhere
\PassOptionsToPackage{unicode,linktoc=all,pdfwindowui,pdfpagemode=FullScreen}{hyperref}
\PassOptionsToPackage{hyphens}{url}
\PassOptionsToPackage{dvipsnames,svgnames,x11names}{xcolor}
%
\documentclass[
  letterpaper,
  paper =a4,
  twoside,
  openright,
  headsepline,
  footsepline,
  listof = totocnumbered,
  chapterprefix = true,
  firstiscover]{scrbook}

\usepackage{amsmath,amssymb}
\usepackage{setspace}
\usepackage{iftex}
\ifPDFTeX
  \usepackage[T1]{fontenc}
  \usepackage[utf8]{inputenc}
  \usepackage{textcomp} % provide euro and other symbols
\else % if luatex or xetex
  \usepackage{unicode-math}
  \defaultfontfeatures{Scale=MatchLowercase}
  \defaultfontfeatures[\rmfamily]{Ligatures=TeX,Scale=1}
\fi
\usepackage{lmodern}
\ifPDFTeX\else  
    % xetex/luatex font selection
\fi
% Use upquote if available, for straight quotes in verbatim environments
\IfFileExists{upquote.sty}{\usepackage{upquote}}{}
\IfFileExists{microtype.sty}{% use microtype if available
  \usepackage[]{microtype}
  \UseMicrotypeSet[protrusion]{basicmath} % disable protrusion for tt fonts
}{}
\makeatletter
\@ifundefined{KOMAClassName}{% if non-KOMA class
  \IfFileExists{parskip.sty}{%
    \usepackage{parskip}
  }{% else
    \setlength{\parindent}{0pt}
    \setlength{\parskip}{6pt plus 2pt minus 1pt}}
}{% if KOMA class
  \KOMAoptions{parskip=half}}
\makeatother
\usepackage{xcolor}
\usepackage[top=1in,bottom=1in,left=1in,right=1in,heightrounded]{geometry}
\setlength{\emergencystretch}{3em} % prevent overfull lines
\setcounter{secnumdepth}{3}
% Make \paragraph and \subparagraph free-standing
\ifx\paragraph\undefined\else
  \let\oldparagraph\paragraph
  \renewcommand{\paragraph}[1]{\oldparagraph{#1}\mbox{}}
\fi
\ifx\subparagraph\undefined\else
  \let\oldsubparagraph\subparagraph
  \renewcommand{\subparagraph}[1]{\oldsubparagraph{#1}\mbox{}}
\fi

\usepackage{color}
\usepackage{fancyvrb}
\newcommand{\VerbBar}{|}
\newcommand{\VERB}{\Verb[commandchars=\\\{\}]}
\DefineVerbatimEnvironment{Highlighting}{Verbatim}{commandchars=\\\{\}}
% Add ',fontsize=\small' for more characters per line
\usepackage{framed}
\definecolor{shadecolor}{RGB}{241,243,245}
\newenvironment{Shaded}{\begin{snugshade}}{\end{snugshade}}
\newcommand{\AlertTok}[1]{\textcolor[rgb]{0.68,0.00,0.00}{#1}}
\newcommand{\AnnotationTok}[1]{\textcolor[rgb]{0.37,0.37,0.37}{#1}}
\newcommand{\AttributeTok}[1]{\textcolor[rgb]{0.40,0.45,0.13}{#1}}
\newcommand{\BaseNTok}[1]{\textcolor[rgb]{0.68,0.00,0.00}{#1}}
\newcommand{\BuiltInTok}[1]{\textcolor[rgb]{0.00,0.23,0.31}{#1}}
\newcommand{\CharTok}[1]{\textcolor[rgb]{0.13,0.47,0.30}{#1}}
\newcommand{\CommentTok}[1]{\textcolor[rgb]{0.37,0.37,0.37}{#1}}
\newcommand{\CommentVarTok}[1]{\textcolor[rgb]{0.37,0.37,0.37}{\textit{#1}}}
\newcommand{\ConstantTok}[1]{\textcolor[rgb]{0.56,0.35,0.01}{#1}}
\newcommand{\ControlFlowTok}[1]{\textcolor[rgb]{0.00,0.23,0.31}{#1}}
\newcommand{\DataTypeTok}[1]{\textcolor[rgb]{0.68,0.00,0.00}{#1}}
\newcommand{\DecValTok}[1]{\textcolor[rgb]{0.68,0.00,0.00}{#1}}
\newcommand{\DocumentationTok}[1]{\textcolor[rgb]{0.37,0.37,0.37}{\textit{#1}}}
\newcommand{\ErrorTok}[1]{\textcolor[rgb]{0.68,0.00,0.00}{#1}}
\newcommand{\ExtensionTok}[1]{\textcolor[rgb]{0.00,0.23,0.31}{#1}}
\newcommand{\FloatTok}[1]{\textcolor[rgb]{0.68,0.00,0.00}{#1}}
\newcommand{\FunctionTok}[1]{\textcolor[rgb]{0.28,0.35,0.67}{#1}}
\newcommand{\ImportTok}[1]{\textcolor[rgb]{0.00,0.46,0.62}{#1}}
\newcommand{\InformationTok}[1]{\textcolor[rgb]{0.37,0.37,0.37}{#1}}
\newcommand{\KeywordTok}[1]{\textcolor[rgb]{0.00,0.23,0.31}{#1}}
\newcommand{\NormalTok}[1]{\textcolor[rgb]{0.00,0.23,0.31}{#1}}
\newcommand{\OperatorTok}[1]{\textcolor[rgb]{0.37,0.37,0.37}{#1}}
\newcommand{\OtherTok}[1]{\textcolor[rgb]{0.00,0.23,0.31}{#1}}
\newcommand{\PreprocessorTok}[1]{\textcolor[rgb]{0.68,0.00,0.00}{#1}}
\newcommand{\RegionMarkerTok}[1]{\textcolor[rgb]{0.00,0.23,0.31}{#1}}
\newcommand{\SpecialCharTok}[1]{\textcolor[rgb]{0.37,0.37,0.37}{#1}}
\newcommand{\SpecialStringTok}[1]{\textcolor[rgb]{0.13,0.47,0.30}{#1}}
\newcommand{\StringTok}[1]{\textcolor[rgb]{0.13,0.47,0.30}{#1}}
\newcommand{\VariableTok}[1]{\textcolor[rgb]{0.07,0.07,0.07}{#1}}
\newcommand{\VerbatimStringTok}[1]{\textcolor[rgb]{0.13,0.47,0.30}{#1}}
\newcommand{\WarningTok}[1]{\textcolor[rgb]{0.37,0.37,0.37}{\textit{#1}}}

\providecommand{\tightlist}{%
  \setlength{\itemsep}{0pt}\setlength{\parskip}{0pt}}\usepackage{longtable,booktabs,array}
\usepackage{calc} % for calculating minipage widths
% Correct order of tables after \paragraph or \subparagraph
\usepackage{etoolbox}
\makeatletter
\patchcmd\longtable{\par}{\if@noskipsec\mbox{}\fi\par}{}{}
\makeatother
% Allow footnotes in longtable head/foot
\IfFileExists{footnotehyper.sty}{\usepackage{footnotehyper}}{\usepackage{footnote}}
\makesavenoteenv{longtable}
\usepackage{graphicx}
\makeatletter
\def\maxwidth{\ifdim\Gin@nat@width>\linewidth\linewidth\else\Gin@nat@width\fi}
\def\maxheight{\ifdim\Gin@nat@height>\textheight\textheight\else\Gin@nat@height\fi}
\makeatother
% Scale images if necessary, so that they will not overflow the page
% margins by default, and it is still possible to overwrite the defaults
% using explicit options in \includegraphics[width, height, ...]{}
\setkeys{Gin}{width=\maxwidth,height=\maxheight,keepaspectratio}
% Set default figure placement to htbp
\makeatletter
\def\fps@figure{htbp}
\makeatother
\newlength{\cslhangindent}
\setlength{\cslhangindent}{1.5em}
\newlength{\csllabelwidth}
\setlength{\csllabelwidth}{3em}
\newlength{\cslentryspacingunit} % times entry-spacing
\setlength{\cslentryspacingunit}{\parskip}
\newenvironment{CSLReferences}[2] % #1 hanging-ident, #2 entry spacing
 {% don't indent paragraphs
  \setlength{\parindent}{0pt}
  % turn on hanging indent if param 1 is 1
  \ifodd #1
  \let\oldpar\par
  \def\par{\hangindent=\cslhangindent\oldpar}
  \fi
  % set entry spacing
  \setlength{\parskip}{#2\cslentryspacingunit}
 }%
 {}
\usepackage{calc}
\newcommand{\CSLBlock}[1]{#1\hfill\break}
\newcommand{\CSLLeftMargin}[1]{\parbox[t]{\csllabelwidth}{#1}}
\newcommand{\CSLRightInline}[1]{\parbox[t]{\linewidth - \csllabelwidth}{#1}\break}
\newcommand{\CSLIndent}[1]{\hspace{\cslhangindent}#1}

\usepackage[automark]{scrlayer-scrpage}%for header and footer
\pagestyle{scrheadings}
\usepackage{pagecolor}%to change the page color of the title page
\usepackage{setspace}%for line spacing
%\doublespacing

\providecommand{\abstractname}{Learning Objectives} % not in scrbook class
\newenvironment{objectives}[1]{%
	\hrule
	\vspace{5pt}
	\small\textbf{\abstractname: } 
	\newline
	\vspace{0.1cm}
	%\small\emph #1     %  emph takes an argument
	\small\emph{#1} % or \small\textit{#1} 
	\itshape % use this if you want the text to be in italics
}{%
	\vspace{5pt}
	\hrule
	\vspace{0.6cm}
}
%Quarto uses \maketitle command by default.
%So, if you want to add your custom title, you will have to
%redefine the command using the \renewcommand.
\definecolor{myblue}{RGB}{70, 130, 180}%steelblue color
\renewcommand{\maketitle}{
\begin{titlepage}
		\pagecolor{myblue}
		\color{white}
	\centering
	\vspace*{\baselineskip}
	\rule{\textwidth}{1.6pt}\vspace*{-\baselineskip}\vspace*{2pt}
	\rule{\textwidth}{0.4pt}\\[\baselineskip]
	{\LARGE Regression Analysis\\ With \\[0.3\baselineskip] R And Easystats}\\[0.2\baselineskip]
	\rule{\textwidth}{0.4pt}\vspace*{-\baselineskip}\vspace{3.2pt}
	\rule{\textwidth}{1.6pt}\\[\baselineskip]
	\scshape
	\vspace*{8\baselineskip}
	Author \\[\baselineskip]
	{\Large Maulik Bhatt \par}
	{\itshape S P Jain Institute of Management \& Research \\ Mumbai\par}
	\vfill
	{\scshape 2023} \\
	{\large Maulik Bhatt}\par%

\end{titlepage}
\nopagecolor%to change the page color of the subsequent pages
}
%redesign chapter and section styles
\definecolor{mybluei}{RGB}{28,138,207}
\definecolor{myblueii}{RGB}{131,197,231}

\addtokomafont{disposition}{\usefont{T1}{qhv}{b}{n}\selectfont\color{myblueii}}

\addtokomafont{chapter}{\fontsize{30pt}{30pt}\selectfont}
\newkomafont{chapternumber}{\fontsize{50}{120}\selectfont\color{white}}
\newkomafont{chaptername}{\itshape\rmfamily\small\color{white}}

\addtokomafont{section}{\fontsize{14pt}{14pt}\selectfont}
\newkomafont{sectionnumber}{\fontsize{18pt}{18pt}\selectfont\rmfamily\color{white}}

\addtokomafont{subsection}{\fontsize{12pt}{12pt}\selectfont}
\newkomafont{subsectionnumber}{\fontsize{16pt}{16pt}\selectfont\rmfamily\color{white}}

\renewcommand\chapterformat{%
  \raisebox{-6pt}{\colorbox{mybluei}{%
    \parbox[b][60pt]{45pt}{\centering%
      {\usekomafont{chaptername}{\chapapp}}%
      \vfill{\usekomafont{chapternumber}{\thechapter\autodot}}%
      \vspace{6pt}%
}}}\enskip}

\renewcommand\sectionformat{%
  \setlength\fboxsep{5pt}%
  \raisebox{-4pt}{\colorbox{mybluei}{%
    \enskip\usekomafont{sectionnumber}{\thesection\autodot}\enskip}%
  \quad%
}}

\renewcommand\subsectionformat{%
  \setlength\fboxsep{5pt}%
  \raisebox{-4pt}{\colorbox{mybluei}{%
    \enskip\usekomafont{subsectionnumber}{\thesubsection\autodot}\enskip}%
  \quad%
}}

\makeatletter
\renewcommand\sectionlinesformat[4]{%
  \makebox[0pt][l]{\rule[-5pt]{\textwidth}{1pt}}%
  \@hangfrom{#3}{#4}%
}
\makeatother
\makeatletter
\makeatother
\makeatletter
\@ifpackageloaded{bookmark}{}{\usepackage{bookmark}}
\makeatother
\makeatletter
\@ifpackageloaded{caption}{}{\usepackage{caption}}
\AtBeginDocument{%
\ifdefined\contentsname
  \renewcommand*\contentsname{Table of contents}
\else
  \newcommand\contentsname{Table of contents}
\fi
\ifdefined\listfigurename
  \renewcommand*\listfigurename{List of Figures}
\else
  \newcommand\listfigurename{List of Figures}
\fi
\ifdefined\listtablename
  \renewcommand*\listtablename{List of Tables}
\else
  \newcommand\listtablename{List of Tables}
\fi
\ifdefined\figurename
  \renewcommand*\figurename{Figure}
\else
  \newcommand\figurename{Figure}
\fi
\ifdefined\tablename
  \renewcommand*\tablename{Table}
\else
  \newcommand\tablename{Table}
\fi
}
\@ifpackageloaded{float}{}{\usepackage{float}}
\floatstyle{ruled}
\@ifundefined{c@chapter}{\newfloat{codelisting}{h}{lop}}{\newfloat{codelisting}{h}{lop}[chapter]}
\floatname{codelisting}{Listing}
\newcommand*\listoflistings{\listof{codelisting}{List of Listings}}
\makeatother
\makeatletter
\@ifpackageloaded{caption}{}{\usepackage{caption}}
\@ifpackageloaded{subcaption}{}{\usepackage{subcaption}}
\makeatother
\makeatletter
\@ifpackageloaded{tcolorbox}{}{\usepackage[skins,breakable]{tcolorbox}}
\makeatother
\makeatletter
\@ifundefined{shadecolor}{\definecolor{shadecolor}{rgb}{.97, .97, .97}}
\makeatother
\makeatletter
\makeatother
\makeatletter
\ifdefined\Shaded\renewenvironment{Shaded}{\begin{tcolorbox}[borderline west={3pt}{0pt}{shadecolor}, frame hidden, interior hidden, enhanced, sharp corners, boxrule=0pt, breakable]}{\end{tcolorbox}}\fi
\makeatother
\makeatletter
\makeatother
\ifLuaTeX
  \usepackage{selnolig}  % disable illegal ligatures
\fi
\IfFileExists{bookmark.sty}{\usepackage{bookmark}}{\usepackage{hyperref}}
\IfFileExists{xurl.sty}{\usepackage{xurl}}{} % add URL line breaks if available
\urlstyle{same} % disable monospaced font for URLs
\hypersetup{
  pdftitle={Regression Analysis With R and Easystats},
  pdfauthor={Maulik Bhatt},
  colorlinks=true,
  linkcolor={Maroon},
  filecolor={Maroon},
  citecolor={Blue},
  urlcolor={Blue},
  pdfcreator={LaTeX via pandoc}}

\title{Regression Analysis With R and Easystats}
\author{Maulik Bhatt}
\date{06/07/2023}

\begin{document}
\frontmatter
\maketitle
\renewcommand*\contentsname{Contents}
{
\hypersetup{linkcolor=}
\setcounter{tocdepth}{2}
\tableofcontents
}
\listoffigures
\listoftables
\setstretch{2}
\mainmatter
\bookmarksetup{startatroot}

\hypertarget{preface}{%
\chapter*{Preface}\label{preface}}
\addcontentsline{toc}{chapter}{Preface}

\markboth{Preface}{Preface}

\bookmarksetup{startatroot}

\hypertarget{acknowledgement}{%
\chapter*{Acknowledgement}\label{acknowledgement}}
\addcontentsline{toc}{chapter}{Acknowledgement}

\markboth{Acknowledgement}{Acknowledgement}

I would like to express my heartfelt gratitude to the individuals who
have played a significant role in the creation of this book on
Regression Analysis with R. Their guidance, knowledge, and support have
been invaluable throughout this journey.

First and foremost, I extend my deepest appreciation to my esteemed
professors, {[}Professor's Name{]}, {[}Professor's Name{]}, and
{[}Professor's Name{]}. Their expertise in statistics and econometrics
has shaped my understanding of quantitative analysis and provided a
solid foundation for this book. Their dedication to teaching and their
unwavering commitment to fostering academic excellence have been
instrumental in my growth as a student of econometrics.

I would like to extend special thanks to my colleague and dear friend,
{[}Colleague's Name{]}. Their introduction to the world of R programming
language opened up a world of possibilities for me. Their patience,
willingness to share their knowledge, and countless hours spent
assisting me with R-related challenges have been indispensable. I am
grateful for their unwavering support and the collaborative environment
we fostered, which significantly enriched my learning experience.

A heartfelt appreciation goes to the developers of R, an open-source
programming language that has revolutionized the field of data analysis
and statistical computing. Their tireless efforts in creating and
continuously improving R have made it an indispensable tool for
researchers and analysts worldwide. Without their dedication, this book
would not have been possible.

I would also like to express my gratitude to the developers of the R
packages that have been instrumental in the analysis and visualization
techniques presented in this book. Their commitment to excellence,
innovation, and user-friendly implementations has immensely contributed
to the field of regression analysis. Their packages have not only
expanded the capabilities of R but have also facilitated the seamless
integration of econometric methodologies into practical applications.

Finally, I extend my heartfelt thanks to my family and friends who have
supported me throughout this writing process. Their encouragement,
understanding, and belief in my abilities have been a constant source of
motivation.

To all those mentioned above, and to anyone else who has contributed to
this book in any way, I offer my deepest appreciation. Your guidance,
knowledge, and support have been invaluable in shaping this work and
have helped me fulfill my goal of sharing the beauty and importance of
regression analysis with R.

Thank you.

{[}Your Name{]}

\bookmarksetup{startatroot}

\hypertarget{introduction}{%
\chapter{Introduction}\label{introduction}}

\begin{longtable}[]{@{}
  >{\centering\arraybackslash}p{(\columnwidth - 0\tabcolsep) * \real{1.0000}}@{}}
\toprule\noalign{}
\begin{minipage}[b]{\linewidth}\centering
Status
\end{minipage} \\
\midrule\noalign{}
\endhead
\bottomrule\noalign{}
\endlastfoot
This chapter is currently a dumping ground for ideas, and we don't
recommend reading it. \\
\end{longtable}

\begin{objectives}{In this chapter, you will learn to}
\begin{itemize}

\item{Importance of regression analysis in econometrics}

\item{Overview of the book's structure and goals}

\item{Introduction to R programming language and its relevance to econometric analysis}

\end{itemize}

\end{objectives}

Econometrics is a combination of two words: Econo + Metric. Econo refers
to concepts of economics, while metric refers to measurement. Let's take
an example of the law of demand from microeconomic theory. We know the
demand of a commodity will decrease if the price of the commodity
increases. But we don't know how much the demand will decrease, given
the increase in the price is one unit (i.e., one Dollar, or one Euro,
etc.). In Econometrics, we measure such increase or decrease using
various experiments. In these experiments, we observe how much the
demand for a commodity increase of decrease in response to a unit change
(increase or decrease) in price of the commodity.

\hypertarget{importance-of-r-for-econometrics-and-statistics}{%
\section{Importance of R for Econometrics and
Statistics}\label{importance-of-r-for-econometrics-and-statistics}}

Many software are available for statistical and econometric analysis.
Some of the popular software include SPSS, SAS, Stata, MATLAB, etc.
These are all proprietary software. Free and open source alternatives
include python and R, which are programming languages. The users need to
type the commands in order to perform various tasks/analysis in these
programming languages.

Python is a general purpose programming langauge, while R is a
statistical programming language. R also has many packages (i.e.,
add-ons, which enhance the functionality of R). As of writing this,
there are \texttt{19697} packages available on CRAN (the Comprehensive R
Archive Network, which is the central authority to decide about R
programming language). Even more packages are available on
\texttt{github}, which is an Internet hosting service for software
development and version control using the version control system Git.
Because of such extensive support, I have chosen R to conduct
econometric analysis for this book.

In particular, I will use \texttt{easystats} set of packages, which are
designed for various statistical analysis tasks. We will see how
\texttt{easystats} make our tasks easier in R.

I will also use \texttt{quarto}, which is a document preparation system,
based on pandoc's markdown. The benefit of \texttt{quarto} is that we
can keep our analysis and prose in the same document. At the end, when
we click on the \texttt{build} button, it renders the entire book in
either pdf or HTML format. It is also possible to render the book in
Microsoft Word format, but this output format has got limited support.
It is more convenient to design and customize the output in HTML and PDF
formats. We will discuss more about this in the chapter of Getting
Started with R.

\bookmarksetup{startatroot}

\hypertarget{introduction-to-econometrics-and-regression-analysis}{%
\chapter{Introduction to Econometrics and Regression
Analysis}\label{introduction-to-econometrics-and-regression-analysis}}

\begin{longtable}[]{@{}
  >{\centering\arraybackslash}p{(\columnwidth - 0\tabcolsep) * \real{1.0000}}@{}}
\toprule\noalign{}
\begin{minipage}[b]{\linewidth}\centering
Status
\end{minipage} \\
\midrule\noalign{}
\endhead
\bottomrule\noalign{}
\endlastfoot
This chapter is currently a dumping ground for ideas, and we don't
recommend reading it. \\
\end{longtable}

\begin{objectives}{In this chapter, you will learn to}
\begin{itemize}

\item{Understanding the basics of econometrics}

\item{Role of regression analysis in econometric modeling}

\item{Overview of the regression analysis process}

\item{Introduction to R for econometric analysis}

\end{itemize}

\end{objectives}

Basics of Econometrics

Role of Regression

Introduction to R

\bookmarksetup{startatroot}

\hypertarget{getting-started-with-r-for-econometrics}{%
\chapter{Getting Started with R for
Econometrics}\label{getting-started-with-r-for-econometrics}}

\begin{longtable}[]{@{}
  >{\centering\arraybackslash}p{(\columnwidth - 0\tabcolsep) * \real{1.0000}}@{}}
\toprule\noalign{}
\begin{minipage}[b]{\linewidth}\centering
Status
\end{minipage} \\
\midrule\noalign{}
\endhead
\bottomrule\noalign{}
\endlastfoot
This chapter is currently a dumping ground for ideas, and we don't
recommend reading it. \\
\end{longtable}

\begin{objectives}{In this chapter, you will learn to}
\begin{itemize}

\item{Introduction to R programming language and its ecosystem}

\item{Setting up the R environment and installing necessary packages}

\item{Loading and manipulating data in R for econometric analysis}

\item{Exploring data visualization techniques using R}

\end{itemize}

\end{objectives}

\hypertarget{a-brief-history-of-r}{%
\section{A Brief History of R}\label{a-brief-history-of-r}}

John Chambers, working at Bell Laboratories, developed S programming
language for statistical analysis. This programming language was
incorporated in the commercial program S-plus. Inspired by this
programming languagage, tw o statistics professors Robert Gentleman and
Ross Ihaka developed a reduced version of S, which they named R.

R was first released in 1993. In 1995, another statistician Martin
Mächler convinced Gentleman and Ihaka to make R source code available to
general public. As a result, R became free and open source under GNU
General Public License.

R also has a robust community of users. As of writing this, there are
19,719 user-developed packages on Comprehensive R Archive Network
(CRAN), which distributes the R source code and packages. Apart from
CRAN, many more packages are also available on code sharing platform
github. Users can download packages from github as well. The Appendix A
explains how to install packages from CRAN and from github.

How to Install R

You can install R from the website of
\href{https://cran.r-project.org/}{CRAN}. Downloadable files and
instructions to download and install R for all the major operating
systems are available on this website. R can be downloaded and installed
on Windows, Mac and Linux platforms.

It is possible to use R right from within this R Graphical User
Interface (R-Gui). But such a work will not be reproducible. It means,
you will be able to work on your system, but you won't be able to share
that work with others. IDE like RStudio makes it much easier to work in
R, and to share your work with others.

How to Install RStudio

Once you have installed R for your operating system, you can visit the
website of \href{https://posit.co/download/rstudio-desktop/}{RStudio}.
Just like R, RStudio is also available on Windows, Mac and Linux
platforms.

Useful Settings in RStudio

I would recommend you not to save the workspace in RStudio. This setting
can be disabled in RStudio as follows:

\begin{figure}[h]

{\centering \includegraphics[width=3.92in,height=\textheight]{images/RStudio-settings.png}

}

\caption{\label{fig-disable-workspace}How to Disable Saving Workspace}

\end{figure}

This setting (of not saving your workspace) seems counter-intuitive. But
it helps you in the longer term. Assume you have to share your work with
someone else. The other person might not have the same settings of
computer in his/her system. So, the work on your system might look
different in that system. Instead, you should save the R Script (or even
better, a Quarto document) in your system, and should share it.

The Concept of Working Directory

R has a concept of working directory. When you save your R script and
your work, it gets saved in the ``working directory'' of R. You can find
and change the working directory using R commands.

\begin{Shaded}
\begin{Highlighting}[numbers=left,,]
\CommentTok{\# Find the working directory}
\FunctionTok{getwd}\NormalTok{()}
\CommentTok{\#Change the working directory}
\FunctionTok{setwd}\NormalTok{(}\StringTok{"Path/to/another/folder"}\NormalTok{)}
\end{Highlighting}
\end{Shaded}

\hypertarget{quarto}{%
\section{Quarto}\label{quarto}}

Quarto is a new open-source scientific and technical publishing system
developed by Posit (the maker of RStudio). Quarto is designed to be
useful to anyone who wants to create reproducible documents. A Quarto
document contains both - the code and the prose. For example, if you are
running regression analysis in R, R Script will only contain your code,
but a Quarto document will contain your R code and your interpretation
of the regression results. Quarto runs computations into separate
pluggable language ``engines'', which helps make this cross-language
functionality easier to support .

Here are some points that emphasize the reproducibility of Quarto over R
scripts:

\begin{enumerate}
\def\labelenumi{\arabic{enumi}.}
\item
  \textbf{Cross-language support}: Quarto is designed to work with
  multiple languages, including Python, bash, Julia, C, SQL, and more.
  This makes it easier to work with different languages in the same
  document.
\item
  \textbf{Built-in output formats}: Quarto generates the output in
  various formats like Microsoft Word, HTML, PDF, beamer, revealjs, etc.
  It also has many options for customizing each format.
\item
  \textbf{Native features for special project types}: Quarto has native
  features for special project types like websites, books, and blogs.
  This means that you don't have to rely on external packages. As a
  matter of fact, this book is written entirely using Quarto in RStudio.
\item
  \textbf{Easier rendering}: Quarto isn't an R package. It's a
  command-line interface that makes it much easier to work with Quarto
  documents outside of the RStudio IDE. You can also use Quarto in other
  IDEs like VS Code.
\end{enumerate}

These features make Quarto a better choice than R scripts when it comes
to reproducibility. With Quarto, you can easily create documents that
are easy to reproduce and share with others.

Optional: Installing \LaTeX

If you want to work in Quarto, and want to generate PDF output,
\LaTeX is required. There are two popular \LaTeX distributions: MiKTeX
and TeX Live. However, I prefer another \LaTeX distribution, called
TinyTeX. It's a light weight distribution of \LaTeX, and works well with
R. You can install it using R as follows:

\begin{Shaded}
\begin{Highlighting}[numbers=left,,]
\CommentTok{\#install tinytex R package}
\FunctionTok{install.packages}\NormalTok{(}\StringTok{"tinytex"}\NormalTok{)}
\CommentTok{\#load it in R}
\FunctionTok{library}\NormalTok{(tinytex)}
\CommentTok{\#use this package to install LaTeX compiler TinyTeX}
\FunctionTok{install\_tinytex}\NormalTok{()}
\end{Highlighting}
\end{Shaded}

These commands install the most commonly used \LaTeX packages into your
system. However, if you want to install all the \LaTeX packages, you can
do so by using \texttt{install\_tinytex(bundle\ =\ "TinyTeX-2")} instead
of \texttt{install\_tinytex()} function above.

\hypertarget{your-first-interaction-with-r}{%
\section{Your First Interaction with
R}\label{your-first-interaction-with-r}}

After installing R and RStudio, you would start RStudio (not R). RStudio
has four parts in the screen. The four parts (called ``panes'') are
source pane, console pane, environment pane and files pane. You can
customize each pane using
\texttt{Tools\ Menu\ \textgreater{}\ Global\ Options\ \textgreater{}\ Pane\ Layout}.
The RStudio screen looks as follows:

\begin{figure}[h]

{\centering \includegraphics[width=4.27in,height=\textheight]{images/RStudio-panes.png}

}

\caption{\label{fig-rstudio-panes}RStudio Panes}

\end{figure}

When you open a new file in RStudio using the
\texttt{Files\ Menu\ \textgreater{}\ New\ File}, you are presented with
many options like R Script, R Markdown Document, Quarto Document, etc. I
would suggest you to choose R Markdown Document or Quarto Document,
because in these two file formats, you can write both - your analysis
and your prose. Your document will be saved in your working directory.

In RStudio's File Menu, you will also find another option:
\texttt{New\ Project}. This is a considerable improvement over a new
file - be it an R Script or a Quarto Document. This is because it
changes the working directory of the project to the folder, where
\texttt{.Rproj} file of the project is stored. You can share this folder
with anyone, and the code will work just fine on that computer also.

When you go to
\texttt{File\ Menu\ \textgreater{}\ New\ Project\ \textgreater{}\ New\ Directory},
you get options as follows:

\begin{figure}[h]

{\centering \includegraphics[width=4.27in,height=\textheight]{images/New_Project.png}

}

\caption{\label{fig-rstudio-project}Types of RStudio Projects}

\end{figure}

As you can see, it has many options like RStudio Project, R Package,
Quarto Project, Quarto Website, Quarto Blog and Quarto Book. This book
is written using a Quarto Book project.

\hypertarget{installing-packages}{%
\section{Installing Packages}\label{installing-packages}}

You can install packages using the command
\texttt{install.packages(package\_name)}. For example, the command to
install the \texttt{easystats} set of packages would be
\texttt{install.packages(easystats)}.

\hypertarget{loading-and-manipulating-data}{%
\section{Loading and Manipulating
Data}\label{loading-and-manipulating-data}}

In order to analyse the data, we need to load or import the data first.
There are many possibilities as far as loading/importing data is
concerned. If the data comes from a package, then you can directly load
the data using the \texttt{data(dataset\_name)} command. For example,
the command \texttt{data(mtcars)} will load the \texttt{mtcars} dataset
into the environment. This dataset comes in-built in R.

Most of the data comes from external sources in different file formats
like csv or Microsoft Excel. Data in csv file format can easily be read
using the \texttt{read.csv(path\_to\_data\_file)} command. There is
\texttt{readr} package also, which is part of the \texttt{tidyverse} set
of packages. This package \texttt{readr} can import csv data and load it
in the environment as a \texttt{tibble}. The native data format in R is
\texttt{data.frame}, and \texttt{tibble} is ``advanced version of
\texttt{data.frame} according to the developers of the tidyverse set of
packages.

Similarly, to import data from Microsoft Excel file, there is
\texttt{readxl} package, which is also a part of \texttt{tidyverse}.
Just like \texttt{readr}, \texttt{readxl} would also import the data in
form of a \texttt{tibble}. Both the packages also share similar
interface to import data.

The packages \texttt{foreign} and \texttt{haven} are useful for
importing data from other statistical packages like SPSS, SAS and STATA.

Another noteworthy package is \texttt{data.table}, which can import data
from many formats. Its function \texttt{fread} is versatile and can read
data in many formats.

\hypertarget{manipulating-data}{%
\subsection{Manipulating Data}\label{manipulating-data}}

Although R is capable of data manipulation without using any other
packages, some of the packages created especially to facilitate data
manipulation makes the process easy and understandable for others. In
our examples, we will use \texttt{tidyverse} set of packages for data
manipulation. In particular, two of its packages \texttt{dplyr} and
\texttt{tidyr} make this process quite easy.

\hypertarget{selecting-columns-from-dataset}{%
\subsection{Selecting Columns from
Dataset}\label{selecting-columns-from-dataset}}

We can use the \texttt{select} command from \texttt{dplyr} for selecting
columns from a dataset.

\begin{Shaded}
\begin{Highlighting}[numbers=left,,]
\FunctionTok{library}\NormalTok{(tidyverse)}
\NormalTok{mtcars }\SpecialCharTok{|\textgreater{}} \FunctionTok{select}\NormalTok{(am}\SpecialCharTok{:}\NormalTok{carb) }\SpecialCharTok{|\textgreater{}}\NormalTok{ knitr}\SpecialCharTok{::}\FunctionTok{kable}\NormalTok{()}
\end{Highlighting}
\end{Shaded}

\begin{longtable}[]{@{}lrrr@{}}
\caption{Selecting Columns from am to carb from mtcars
dataset}\tabularnewline
\toprule\noalign{}
& am & gear & carb \\
\midrule\noalign{}
\endfirsthead
\toprule\noalign{}
& am & gear & carb \\
\midrule\noalign{}
\endhead
\bottomrule\noalign{}
\endlastfoot
Mazda RX4 & 1 & 4 & 4 \\
Mazda RX4 Wag & 1 & 4 & 4 \\
Datsun 710 & 1 & 4 & 1 \\
Hornet 4 Drive & 0 & 3 & 1 \\
Hornet Sportabout & 0 & 3 & 2 \\
Valiant & 0 & 3 & 1 \\
Duster 360 & 0 & 3 & 4 \\
Merc 240D & 0 & 4 & 2 \\
Merc 230 & 0 & 4 & 2 \\
Merc 280 & 0 & 4 & 4 \\
Merc 280C & 0 & 4 & 4 \\
Merc 450SE & 0 & 3 & 3 \\
Merc 450SL & 0 & 3 & 3 \\
Merc 450SLC & 0 & 3 & 3 \\
Cadillac Fleetwood & 0 & 3 & 4 \\
Lincoln Continental & 0 & 3 & 4 \\
Chrysler Imperial & 0 & 3 & 4 \\
Fiat 128 & 1 & 4 & 1 \\
Honda Civic & 1 & 4 & 2 \\
Toyota Corolla & 1 & 4 & 1 \\
Toyota Corona & 0 & 3 & 1 \\
Dodge Challenger & 0 & 3 & 2 \\
AMC Javelin & 0 & 3 & 2 \\
Camaro Z28 & 0 & 3 & 4 \\
Pontiac Firebird & 0 & 3 & 2 \\
Fiat X1-9 & 1 & 4 & 1 \\
Porsche 914-2 & 1 & 5 & 2 \\
Lotus Europa & 1 & 5 & 2 \\
Ford Pantera L & 1 & 5 & 4 \\
Ferrari Dino & 1 & 5 & 6 \\
Maserati Bora & 1 & 5 & 8 \\
Volvo 142E & 1 & 4 & 2 \\
\end{longtable}

In the above command, we first loaded the \texttt{tidyverse} set of
packages, so that its functionality is available in the R session. Then
we took the built-in dataset \texttt{mtcars}, and then selected the
columns from \texttt{am} to \texttt{carb} in the dataset. The
\texttt{\textbar{}\textgreater{}} is called the \texttt{pipe\ operator}
in R. It was introduced in R version 4.1. Before that, there was another
pipe operator (\texttt{\%\textgreater{}\%}) in the \texttt{magrittr}
package, which is also a part of \texttt{tidyverse}. To understand how
pipe operator makes coding easy and readable, imagine you want to take
g(f(x)), which reads g of f of x. It means you want to take a value x,
apply the function f on that value, and on the output, you want to apply
another function g. Before the introduction of pipe operator, you would
have written it as \texttt{g(f(x))}, but now after R 4.1, you can write
it as
\texttt{x\ \textbar{}\textgreater{}\ f()\ \textbar{}\textgreater{}\ g()}.

\hypertarget{selecting-rows-from-a-dataset-aka-filtering}{%
\subsection{\texorpdfstring{Selecting Rows from a Dataset (aka
\texttt{filtering})}{Selecting Rows from a Dataset (aka filtering)}}\label{selecting-rows-from-a-dataset-aka-filtering}}

We can use \texttt{filter} function from \texttt{dplyr} for filtering
rows of a dataset. For example, we want to filter the rows of
\texttt{mtcars} dataset, where \texttt{cyl} is equal to 6. We can do
this by

\begin{Shaded}
\begin{Highlighting}[numbers=left,,]
\NormalTok{mtcars }\SpecialCharTok{|\textgreater{}} \FunctionTok{filter}\NormalTok{(cyl}\SpecialCharTok{==}\DecValTok{6}\NormalTok{) }\SpecialCharTok{|\textgreater{}}\NormalTok{ knitr}\SpecialCharTok{::}\FunctionTok{kable}\NormalTok{()}
\end{Highlighting}
\end{Shaded}

\begin{longtable}[]{@{}
  >{\raggedright\arraybackslash}p{(\columnwidth - 22\tabcolsep) * \real{0.2239}}
  >{\raggedleft\arraybackslash}p{(\columnwidth - 22\tabcolsep) * \real{0.0746}}
  >{\raggedleft\arraybackslash}p{(\columnwidth - 22\tabcolsep) * \real{0.0597}}
  >{\raggedleft\arraybackslash}p{(\columnwidth - 22\tabcolsep) * \real{0.0896}}
  >{\raggedleft\arraybackslash}p{(\columnwidth - 22\tabcolsep) * \real{0.0597}}
  >{\raggedleft\arraybackslash}p{(\columnwidth - 22\tabcolsep) * \real{0.0746}}
  >{\raggedleft\arraybackslash}p{(\columnwidth - 22\tabcolsep) * \real{0.0896}}
  >{\raggedleft\arraybackslash}p{(\columnwidth - 22\tabcolsep) * \real{0.0896}}
  >{\raggedleft\arraybackslash}p{(\columnwidth - 22\tabcolsep) * \real{0.0448}}
  >{\raggedleft\arraybackslash}p{(\columnwidth - 22\tabcolsep) * \real{0.0448}}
  >{\raggedleft\arraybackslash}p{(\columnwidth - 22\tabcolsep) * \real{0.0746}}
  >{\raggedleft\arraybackslash}p{(\columnwidth - 22\tabcolsep) * \real{0.0746}}@{}}
\caption{Selecting Rows from mtcars dataset where value of cyl is
6}\tabularnewline
\toprule\noalign{}
\begin{minipage}[b]{\linewidth}\raggedright
\end{minipage} & \begin{minipage}[b]{\linewidth}\raggedleft
mpg
\end{minipage} & \begin{minipage}[b]{\linewidth}\raggedleft
cyl
\end{minipage} & \begin{minipage}[b]{\linewidth}\raggedleft
disp
\end{minipage} & \begin{minipage}[b]{\linewidth}\raggedleft
hp
\end{minipage} & \begin{minipage}[b]{\linewidth}\raggedleft
drat
\end{minipage} & \begin{minipage}[b]{\linewidth}\raggedleft
wt
\end{minipage} & \begin{minipage}[b]{\linewidth}\raggedleft
qsec
\end{minipage} & \begin{minipage}[b]{\linewidth}\raggedleft
vs
\end{minipage} & \begin{minipage}[b]{\linewidth}\raggedleft
am
\end{minipage} & \begin{minipage}[b]{\linewidth}\raggedleft
gear
\end{minipage} & \begin{minipage}[b]{\linewidth}\raggedleft
carb
\end{minipage} \\
\midrule\noalign{}
\endfirsthead
\toprule\noalign{}
\begin{minipage}[b]{\linewidth}\raggedright
\end{minipage} & \begin{minipage}[b]{\linewidth}\raggedleft
mpg
\end{minipage} & \begin{minipage}[b]{\linewidth}\raggedleft
cyl
\end{minipage} & \begin{minipage}[b]{\linewidth}\raggedleft
disp
\end{minipage} & \begin{minipage}[b]{\linewidth}\raggedleft
hp
\end{minipage} & \begin{minipage}[b]{\linewidth}\raggedleft
drat
\end{minipage} & \begin{minipage}[b]{\linewidth}\raggedleft
wt
\end{minipage} & \begin{minipage}[b]{\linewidth}\raggedleft
qsec
\end{minipage} & \begin{minipage}[b]{\linewidth}\raggedleft
vs
\end{minipage} & \begin{minipage}[b]{\linewidth}\raggedleft
am
\end{minipage} & \begin{minipage}[b]{\linewidth}\raggedleft
gear
\end{minipage} & \begin{minipage}[b]{\linewidth}\raggedleft
carb
\end{minipage} \\
\midrule\noalign{}
\endhead
\bottomrule\noalign{}
\endlastfoot
Mazda RX4 & 21.0 & 6 & 160.0 & 110 & 3.90 & 2.620 & 16.46 & 0 & 1 & 4 &
4 \\
Mazda RX4 Wag & 21.0 & 6 & 160.0 & 110 & 3.90 & 2.875 & 17.02 & 0 & 1 &
4 & 4 \\
Hornet 4 Drive & 21.4 & 6 & 258.0 & 110 & 3.08 & 3.215 & 19.44 & 1 & 0 &
3 & 1 \\
Valiant & 18.1 & 6 & 225.0 & 105 & 2.76 & 3.460 & 20.22 & 1 & 0 & 3 &
1 \\
Merc 280 & 19.2 & 6 & 167.6 & 123 & 3.92 & 3.440 & 18.30 & 1 & 0 & 4 &
4 \\
Merc 280C & 17.8 & 6 & 167.6 & 123 & 3.92 & 3.440 & 18.90 & 1 & 0 & 4 &
4 \\
Ferrari Dino & 19.7 & 6 & 145.0 & 175 & 3.62 & 2.770 & 15.50 & 0 & 1 & 5
& 6 \\
\end{longtable}

We can see that in this code, we have used ``='' twice. In R, ``='' is
an assignment operator. It means, when we write \texttt{cyl=6}, it
assigns value 6 to the variable \texttt{cyl}. So, when we want to check
equality, we have to use ``=='' instead of ``=''.

\hypertarget{arrange-data}{%
\subsection{Arrange Data}\label{arrange-data}}

Using \texttt{dplyr}, it is easy to arrange or sort data in ascending or
descending order of a column: use the \texttt{arrange} function for
that.

\begin{Shaded}
\begin{Highlighting}[numbers=left,,]
\NormalTok{mtcars }\SpecialCharTok{|\textgreater{}} \FunctionTok{arrange}\NormalTok{(mpg)}\SpecialCharTok{|\textgreater{}} \FunctionTok{select}\NormalTok{(mpg}\SpecialCharTok{:}\NormalTok{disp) }\SpecialCharTok{|\textgreater{}}\NormalTok{ knitr}\SpecialCharTok{::}\FunctionTok{kable}\NormalTok{()}
\end{Highlighting}
\end{Shaded}

\begin{longtable}[]{@{}lrrr@{}}
\caption{Arrange the Data in Ascending Order of
\texttt{mpg}}\tabularnewline
\toprule\noalign{}
& mpg & cyl & disp \\
\midrule\noalign{}
\endfirsthead
\toprule\noalign{}
& mpg & cyl & disp \\
\midrule\noalign{}
\endhead
\bottomrule\noalign{}
\endlastfoot
Cadillac Fleetwood & 10.4 & 8 & 472.0 \\
Lincoln Continental & 10.4 & 8 & 460.0 \\
Camaro Z28 & 13.3 & 8 & 350.0 \\
Duster 360 & 14.3 & 8 & 360.0 \\
Chrysler Imperial & 14.7 & 8 & 440.0 \\
Maserati Bora & 15.0 & 8 & 301.0 \\
Merc 450SLC & 15.2 & 8 & 275.8 \\
AMC Javelin & 15.2 & 8 & 304.0 \\
Dodge Challenger & 15.5 & 8 & 318.0 \\
Ford Pantera L & 15.8 & 8 & 351.0 \\
Merc 450SE & 16.4 & 8 & 275.8 \\
Merc 450SL & 17.3 & 8 & 275.8 \\
Merc 280C & 17.8 & 6 & 167.6 \\
Valiant & 18.1 & 6 & 225.0 \\
Hornet Sportabout & 18.7 & 8 & 360.0 \\
Merc 280 & 19.2 & 6 & 167.6 \\
Pontiac Firebird & 19.2 & 8 & 400.0 \\
Ferrari Dino & 19.7 & 6 & 145.0 \\
Mazda RX4 & 21.0 & 6 & 160.0 \\
Mazda RX4 Wag & 21.0 & 6 & 160.0 \\
Hornet 4 Drive & 21.4 & 6 & 258.0 \\
Volvo 142E & 21.4 & 4 & 121.0 \\
Toyota Corona & 21.5 & 4 & 120.1 \\
Datsun 710 & 22.8 & 4 & 108.0 \\
Merc 230 & 22.8 & 4 & 140.8 \\
Merc 240D & 24.4 & 4 & 146.7 \\
Porsche 914-2 & 26.0 & 4 & 120.3 \\
Fiat X1-9 & 27.3 & 4 & 79.0 \\
Honda Civic & 30.4 & 4 & 75.7 \\
Lotus Europa & 30.4 & 4 & 95.1 \\
Fiat 128 & 32.4 & 4 & 78.7 \\
Toyota Corolla & 33.9 & 4 & 71.1 \\
\end{longtable}

You can similarly arrange the data in descending order of \texttt{mpg}
column in the \texttt{mtcars} dataset.

\begin{Shaded}
\begin{Highlighting}[numbers=left,,]
\NormalTok{mtcars }\SpecialCharTok{|\textgreater{}} \FunctionTok{arrange}\NormalTok{(}\FunctionTok{desc}\NormalTok{(mpg)) }\SpecialCharTok{|\textgreater{}} \FunctionTok{select}\NormalTok{(mpg}\SpecialCharTok{:}\NormalTok{disp) }\SpecialCharTok{|\textgreater{}}\NormalTok{ knitr}\SpecialCharTok{::}\FunctionTok{kable}\NormalTok{()}
\end{Highlighting}
\end{Shaded}

\begin{longtable}[]{@{}lrrr@{}}
\caption{Arrange the Data in Descending Order of
\texttt{mpg}}\tabularnewline
\toprule\noalign{}
& mpg & cyl & disp \\
\midrule\noalign{}
\endfirsthead
\toprule\noalign{}
& mpg & cyl & disp \\
\midrule\noalign{}
\endhead
\bottomrule\noalign{}
\endlastfoot
Toyota Corolla & 33.9 & 4 & 71.1 \\
Fiat 128 & 32.4 & 4 & 78.7 \\
Honda Civic & 30.4 & 4 & 75.7 \\
Lotus Europa & 30.4 & 4 & 95.1 \\
Fiat X1-9 & 27.3 & 4 & 79.0 \\
Porsche 914-2 & 26.0 & 4 & 120.3 \\
Merc 240D & 24.4 & 4 & 146.7 \\
Datsun 710 & 22.8 & 4 & 108.0 \\
Merc 230 & 22.8 & 4 & 140.8 \\
Toyota Corona & 21.5 & 4 & 120.1 \\
Hornet 4 Drive & 21.4 & 6 & 258.0 \\
Volvo 142E & 21.4 & 4 & 121.0 \\
Mazda RX4 & 21.0 & 6 & 160.0 \\
Mazda RX4 Wag & 21.0 & 6 & 160.0 \\
Ferrari Dino & 19.7 & 6 & 145.0 \\
Merc 280 & 19.2 & 6 & 167.6 \\
Pontiac Firebird & 19.2 & 8 & 400.0 \\
Hornet Sportabout & 18.7 & 8 & 360.0 \\
Valiant & 18.1 & 6 & 225.0 \\
Merc 280C & 17.8 & 6 & 167.6 \\
Merc 450SL & 17.3 & 8 & 275.8 \\
Merc 450SE & 16.4 & 8 & 275.8 \\
Ford Pantera L & 15.8 & 8 & 351.0 \\
Dodge Challenger & 15.5 & 8 & 318.0 \\
Merc 450SLC & 15.2 & 8 & 275.8 \\
AMC Javelin & 15.2 & 8 & 304.0 \\
Maserati Bora & 15.0 & 8 & 301.0 \\
Chrysler Imperial & 14.7 & 8 & 440.0 \\
Duster 360 & 14.3 & 8 & 360.0 \\
Camaro Z28 & 13.3 & 8 & 350.0 \\
Cadillac Fleetwood & 10.4 & 8 & 472.0 \\
Lincoln Continental & 10.4 & 8 & 460.0 \\
\end{longtable}

\hypertarget{create-a-new-variable}{%
\subsection{Create a New Variable}\label{create-a-new-variable}}

You can create a new variable using the \texttt{mutate} function in
\texttt{dplyr} package. For example, you want to create a new variable,
called FuelEff, which is defined as reciprocal of mpg, then you can do
it as follows:

\begin{Shaded}
\begin{Highlighting}[numbers=left,,]
\NormalTok{mtcars }\SpecialCharTok{|\textgreater{}} \FunctionTok{mutate}\NormalTok{(}\AttributeTok{FuelEff =} \DecValTok{1}\SpecialCharTok{/}\NormalTok{mpg) }\SpecialCharTok{|\textgreater{}} \FunctionTok{select}\NormalTok{(am}\SpecialCharTok{:}\NormalTok{FuelEff) }\SpecialCharTok{|\textgreater{}}\NormalTok{ knitr}\SpecialCharTok{::}\FunctionTok{kable}\NormalTok{()}
\end{Highlighting}
\end{Shaded}

\begin{longtable}[]{@{}lrrrr@{}}
\caption{Create a New Variable in \texttt{mtcars}
dataset}\tabularnewline
\toprule\noalign{}
& am & gear & carb & FuelEff \\
\midrule\noalign{}
\endfirsthead
\toprule\noalign{}
& am & gear & carb & FuelEff \\
\midrule\noalign{}
\endhead
\bottomrule\noalign{}
\endlastfoot
Mazda RX4 & 1 & 4 & 4 & 0.0476190 \\
Mazda RX4 Wag & 1 & 4 & 4 & 0.0476190 \\
Datsun 710 & 1 & 4 & 1 & 0.0438596 \\
Hornet 4 Drive & 0 & 3 & 1 & 0.0467290 \\
Hornet Sportabout & 0 & 3 & 2 & 0.0534759 \\
Valiant & 0 & 3 & 1 & 0.0552486 \\
Duster 360 & 0 & 3 & 4 & 0.0699301 \\
Merc 240D & 0 & 4 & 2 & 0.0409836 \\
Merc 230 & 0 & 4 & 2 & 0.0438596 \\
Merc 280 & 0 & 4 & 4 & 0.0520833 \\
Merc 280C & 0 & 4 & 4 & 0.0561798 \\
Merc 450SE & 0 & 3 & 3 & 0.0609756 \\
Merc 450SL & 0 & 3 & 3 & 0.0578035 \\
Merc 450SLC & 0 & 3 & 3 & 0.0657895 \\
Cadillac Fleetwood & 0 & 3 & 4 & 0.0961538 \\
Lincoln Continental & 0 & 3 & 4 & 0.0961538 \\
Chrysler Imperial & 0 & 3 & 4 & 0.0680272 \\
Fiat 128 & 1 & 4 & 1 & 0.0308642 \\
Honda Civic & 1 & 4 & 2 & 0.0328947 \\
Toyota Corolla & 1 & 4 & 1 & 0.0294985 \\
Toyota Corona & 0 & 3 & 1 & 0.0465116 \\
Dodge Challenger & 0 & 3 & 2 & 0.0645161 \\
AMC Javelin & 0 & 3 & 2 & 0.0657895 \\
Camaro Z28 & 0 & 3 & 4 & 0.0751880 \\
Pontiac Firebird & 0 & 3 & 2 & 0.0520833 \\
Fiat X1-9 & 1 & 4 & 1 & 0.0366300 \\
Porsche 914-2 & 1 & 5 & 2 & 0.0384615 \\
Lotus Europa & 1 & 5 & 2 & 0.0328947 \\
Ford Pantera L & 1 & 5 & 4 & 0.0632911 \\
Ferrari Dino & 1 & 5 & 6 & 0.0507614 \\
Maserati Bora & 1 & 5 & 8 & 0.0666667 \\
Volvo 142E & 1 & 4 & 2 & 0.0467290 \\
\end{longtable}

Summarize a Data

Suppose you want to see what is the average \texttt{mpg} in the whole
dataset? Remember mpg stands for miles per gallon. So, this column shows
the fuel efficiency of various cars. So, you can use \texttt{summary}
function from dplyr to get this answer.

\begin{Shaded}
\begin{Highlighting}[numbers=left,,]
\NormalTok{mtcars }\SpecialCharTok{|\textgreater{}} \FunctionTok{summarise}\NormalTok{(}\StringTok{"Average MPG"} \OtherTok{=} \FunctionTok{mean}\NormalTok{(mpg)) }\SpecialCharTok{|\textgreater{}}\NormalTok{ knitr}\SpecialCharTok{::}\FunctionTok{kable}\NormalTok{()}
\end{Highlighting}
\end{Shaded}

\begin{longtable}[]{@{}r@{}}
\caption{Average Miles Per Gallon of Cars in USA}\tabularnewline
\toprule\noalign{}
Average MPG \\
\midrule\noalign{}
\endfirsthead
\toprule\noalign{}
Average MPG \\
\midrule\noalign{}
\endhead
\bottomrule\noalign{}
\endlastfoot
20.09062 \\
\end{longtable}

However, this function \texttt{summarise} becomes more powerful when
combined with another function \texttt{group\_by}. For example, you want
to know if there is difference in mean mpg for different level of cyl.
You can see it using the following code:

\begin{Shaded}
\begin{Highlighting}[numbers=left,,]
\NormalTok{mtcars }\SpecialCharTok{|\textgreater{}} \FunctionTok{group\_by}\NormalTok{(cyl) }\SpecialCharTok{|\textgreater{}} \FunctionTok{summarise}\NormalTok{(}\StringTok{"Average MPG"} \OtherTok{=} \FunctionTok{mean}\NormalTok{(mpg)) }\SpecialCharTok{|\textgreater{}}\NormalTok{ knitr}\SpecialCharTok{::}\FunctionTok{kable}\NormalTok{()}
\end{Highlighting}
\end{Shaded}

\begin{longtable}[]{@{}rr@{}}
\caption{Miles Per Gallon for Different Values of
Cylinder}\tabularnewline
\toprule\noalign{}
cyl & Average MPG \\
\midrule\noalign{}
\endfirsthead
\toprule\noalign{}
cyl & Average MPG \\
\midrule\noalign{}
\endhead
\bottomrule\noalign{}
\endlastfoot
4 & 26.66364 \\
6 & 19.74286 \\
8 & 15.10000 \\
\end{longtable}

\bookmarksetup{startatroot}

\hypertarget{simple-linear-regression}{%
\chapter{Simple Linear Regression}\label{simple-linear-regression}}

\begin{longtable}[]{@{}
  >{\centering\arraybackslash}p{(\columnwidth - 0\tabcolsep) * \real{1.0000}}@{}}
\toprule\noalign{}
\begin{minipage}[b]{\linewidth}\centering
Status
\end{minipage} \\
\midrule\noalign{}
\endhead
\bottomrule\noalign{}
\endlastfoot
This chapter is currently a dumping ground for ideas, and we don't
recommend reading it. \\
\end{longtable}

\begin{objectives}{In this chapter, you will learn to}
\begin{itemize}

\item{Understanding the principles of simple linear regression}

\item{Performing simple linear regression in R for econometric analysis}

\item{Interpreting regression results in the context of economic variables}

\item{Assessing model assumptions and addressing violations}

\item{Practical examples and exercises using R}

\end{itemize}

\end{objectives}

\begin{Shaded}
\begin{Highlighting}[numbers=left,,]
\FunctionTok{library}\NormalTok{(easystats)}
\FunctionTok{library}\NormalTok{(ggplot2)}
\end{Highlighting}
\end{Shaded}

Introduction

Linear regression is among the fundamental concepts in Econometrics. In
linear regression, we try to estimate how much one variable will change,
with response to a change in another variable. The controlled variable
is called predictor or independent variable. The variable, which changes
as a response to a change in the controlled variable is called response
or dependent variable. In R, we denote such relationship with
\texttt{y\textasciitilde{}x}, where y is the response and x is the
predictor. The notation \texttt{y\textasciitilde{}x} is read as ``y
explained by x''.

In mathematical terms, we are fitting a linear equation between x and y.
When we write \texttt{y\textasciitilde{}x}, it means
\texttt{y\ =\ ax\ +\ b}, and we need to find a and b from the data
points given.

Let's understand what it means for us. The workflow for the linear
regression problem would be:

\begin{enumerate}
\def\labelenumi{\arabic{enumi}.}
\item
  We would be given some observations of both - independent variable and
  dependent variable.
\item
  we graph these data points, using a coordinate system (like Cartesian
  system). Each value is represented by a dot. Such a diagram is called
  a scatter-plot diagram.
\item
  After graphing the points onto a scatter plot diagram, linear
  regression analysis seeks to find the best-fit line to fit the points
  as closely as possible.
\item
  This best-fit line is a line, which minimizes the distance between the
  points falling above or below the lines.
\end{enumerate}

\href{https://www.fiverr.com/resources/guides/data/linear-regression-101}{Linear
Regression 101: What Is It And How Is It Done? \textbar{} Fiverr}

Principles of Linear Regression (Gauss Markov)

Example in R

For understanding regression, we can generate a dummy dataset, and
create a regression model on the basis of the data.

\begin{Shaded}
\begin{Highlighting}[numbers=left,,]
\NormalTok{dummy }\OtherTok{\textless{}{-}} \FunctionTok{data.frame}\NormalTok{(}\AttributeTok{y =} \FunctionTok{rnorm}\NormalTok{(}\DecValTok{100}\NormalTok{),}\AttributeTok{x =} \FunctionTok{rpois}\NormalTok{(}\DecValTok{100}\NormalTok{,}\DecValTok{3}\NormalTok{))}
\end{Highlighting}
\end{Shaded}

In the code above, we have created a dummy data. The \texttt{rnorm(n)}
function generates n number of data points based on normal distribution.
The default mean and standard deviation are 0 and 1 respectively.
Similarly, the \texttt{rpois(n,l)} function generates n number of data
points based on poisson distribution, with \lambda = l. We can plot the
data to explore it visually. We will use \texttt{ggplot2} package to
create this scatterplot diagram.

\begin{Shaded}
\begin{Highlighting}[numbers=left,,]
\FunctionTok{ggplot}\NormalTok{(dummy)}\SpecialCharTok{+}\FunctionTok{aes}\NormalTok{(x,y)}\SpecialCharTok{+}\FunctionTok{geom\_point}\NormalTok{()}\SpecialCharTok{+}\FunctionTok{labs}\NormalTok{(}\AttributeTok{title =} \StringTok{"Scatterplot Diagram of X and Y"}\NormalTok{)}
\end{Highlighting}
\end{Shaded}

\begin{figure}[H]

{\centering \includegraphics{lm_files/figure-pdf/fig-scatterplot-1.pdf}

}

\caption{\label{fig-scatterplot}Scatterplot of X and Y}

\end{figure}

Now we can create a regression model from this dataset.

\begin{Shaded}
\begin{Highlighting}[numbers=left,,]
\NormalTok{model1 }\OtherTok{\textless{}{-}} \FunctionTok{lm}\NormalTok{(y}\SpecialCharTok{\textasciitilde{}}\NormalTok{x, }\AttributeTok{data =}\NormalTok{ dummy)}
\end{Highlighting}
\end{Shaded}

This model stores a lot of information, but it is difficult to
understand and get the meaning out of it. So, one way is to use the
\texttt{summary} function on this \texttt{model1} object.

\begin{Shaded}
\begin{Highlighting}[numbers=left,,]
\FunctionTok{summary}\NormalTok{(model1)}
\end{Highlighting}
\end{Shaded}

\begin{verbatim}

Call:
lm(formula = y ~ x, data = dummy)

Residuals:
    Min      1Q  Median      3Q     Max 
-2.6342 -0.4445  0.0324  0.6090  2.7336 

Coefficients:
            Estimate Std. Error t value Pr(>|t|)
(Intercept)  0.24693    0.20632   1.197    0.234
x           -0.05627    0.05672  -0.992    0.324

Residual standard error: 1.051 on 98 degrees of freedom
Multiple R-squared:  0.009944,  Adjusted R-squared:  -0.0001589 
F-statistic: 0.9843 on 1 and 98 DF,  p-value: 0.3236
\end{verbatim}

The problem with this summary object is that it is not a data.frame. It
is difficult to put this object in a research paper or any other
academic submisison. So, we can use the \texttt{easystats} packages for
this. The \texttt{model\_parameters} function would show us the
parameters (aka coefficients) of the model, and the
\texttt{model\_performance} function will show us the effectiveness of
the regression model. We can also use the \texttt{display} function to
beautify the table in the output.

\begin{Shaded}
\begin{Highlighting}[numbers=left,,]
\NormalTok{model1 }\SpecialCharTok{|\textgreater{}} \FunctionTok{model\_parameters}\NormalTok{() }\SpecialCharTok{|\textgreater{}} 
  \FunctionTok{display}\NormalTok{(}\AttributeTok{format =} \StringTok{"markdown"}\NormalTok{, }\AttributeTok{caption =} \StringTok{"Regression Parameters"}\NormalTok{)}
\end{Highlighting}
\end{Shaded}

\hypertarget{tbl-regression-params}{}
\begin{longtable}[]{@{}lccccc@{}}
\caption{\label{tbl-regression-params}Regression
Parameters}\tabularnewline
\toprule\noalign{}
Parameter & Coefficient & SE & 95\% CI & t(98) & p \\
\midrule\noalign{}
\endfirsthead
\toprule\noalign{}
Parameter & Coefficient & SE & 95\% CI & t(98) & p \\
\midrule\noalign{}
\endhead
\bottomrule\noalign{}
\endlastfoot
(Intercept) & 0.25 & 0.21 & (-0.16, 0.66) & 1.20 & 0.234 \\
x & -0.06 & 0.06 & (-0.17, 0.06) & -0.99 & 0.324 \\
\end{longtable}

\begin{Shaded}
\begin{Highlighting}[numbers=left,,]
\NormalTok{model1 }\SpecialCharTok{|\textgreater{}} \FunctionTok{model\_performance}\NormalTok{() }\SpecialCharTok{|\textgreater{}} 
  \FunctionTok{display}\NormalTok{(}\AttributeTok{format =} \StringTok{"markdown"}\NormalTok{, }\AttributeTok{caption =} \StringTok{"Regression Effectiveness"}\NormalTok{)}
\end{Highlighting}
\end{Shaded}

\hypertarget{tbl-regression-effectiveness}{}
\begin{longtable}[]{@{}lcccccc@{}}
\caption{\label{tbl-regression-effectiveness}Regression
Effectiveness}\tabularnewline
\toprule\noalign{}
AIC & AICc & BIC & R2 & R2 (adj.) & RMSE & Sigma \\
\midrule\noalign{}
\endfirsthead
\toprule\noalign{}
AIC & AICc & BIC & R2 & R2 (adj.) & RMSE & Sigma \\
\midrule\noalign{}
\endhead
\bottomrule\noalign{}
\endlastfoot
297.71 & 297.96 & 305.53 & 9.94e-03 & -1.59e-04 & 1.04 & 1.05 \\
\end{longtable}

\hypertarget{interpretation-of-regression-model}{%
\subsection{Interpretation of Regression
Model}\label{interpretation-of-regression-model}}

- The intercept is statistically non-significant and negative (beta =
-0.16, 95\% CI {[}-0.59, 0.27{]}, t(98) = -0.74, p = 0.461; Std. beta =
-5.50e-18, 95\% CI {[}-0.20, 0.20{]}).

\begin{itemize}
\tightlist
\item
  The effect of x is statistically non-significant and positive (beta =
  0.07, 95\% CI {[}-0.04, 0.17{]}, t(98) = 1.24, p = 0.219; Std. beta =
  0.12, 95\% CI {[}-0.07, 0.32{]}).
\end{itemize}

The model explains a statistically not significant and very

weak proportion of variance (R\^{}2 = 0.02, F(1, 98) = 1.53, p =

0.219, adj. R\^{}2 = 5.32e-03).

\hypertarget{non-linear-relationship-in-regression}{%
\subsection{Non-Linear Relationship in
Regression}\label{non-linear-relationship-in-regression}}

Sometimes, the relationship between the dependent and independent
variable is not linear, but quadratic. This can be understood either
from the scatter plot diagram or from theoretical considerations. In
mathematical terms, the relationship is

\begin{equation}\protect\hypertarget{eq-quadratic}{}{
Y = \beta_0 + \beta_1*X^2 + \gamma
}\label{eq-quadratic}\end{equation}

where X and Y are independent and dependent variables, respectively.
This is still considered ``Linear Regression'', because we are
interested in the the linearity of the coefficient term (i.e., \beta),
not the linearity of the variables. In the equation above, both the
coefficients (\(\beta_0\) and \(\beta_1\)) are linear, and so is the
error term, \(\gamma\) in this equation.

\bookmarksetup{startatroot}

\hypertarget{multiple-linear-regression}{%
\chapter{Multiple Linear Regression}\label{multiple-linear-regression}}

\begin{longtable}[]{@{}
  >{\centering\arraybackslash}p{(\columnwidth - 0\tabcolsep) * \real{1.0000}}@{}}
\toprule\noalign{}
\begin{minipage}[b]{\linewidth}\centering
Status
\end{minipage} \\
\midrule\noalign{}
\endhead
\bottomrule\noalign{}
\endlastfoot
This chapter is currently a dumping ground for ideas, and we don't
recommend reading it. \\
\end{longtable}

\begin{objectives}{In this chapter, you will learn to}
\begin{itemize}

\item{Extending regression analysis to multiple independent variables}

\item{Building and interpreting multiple linear regression models in R}

\item{Handling multicollinearity and selecting significant predictors in an economic context}

\item{Model evaluation and diagnostics in econometric regression}

\item{Application of multiple linear regression in economic analysis using R}

\end{itemize}

\end{objectives}

\hypertarget{introduction-1}{%
\subsection{Introduction}\label{introduction-1}}

In the previous chapter, we saw simple linear regression, in which we
attempted to understand the relationship between two variables - one
called independent variable and the other called the dependent variable.
Multiple linear regression is a statistical technique that allows us to
analyze the relationship between two or more independent variables and a
dependent variable. It is an extension of simple linear regression,
which only involves one independent variable. Multiple linear regression
is used to predict the value of a dependent variable based on the values
of two or more independent variables. It is commonly used in fields such
as economics, finance, and social sciences.

Assumptions of Multiple Linear Regression

There are a few assumptions of multiple linear regression. They are as
follows:

\begin{enumerate}
\def\labelenumi{\arabic{enumi}.}
\item
  Linearity of coefficients and error terms

  The coefficients (\(\beta\)) and the error terms (\(\epsilon\)) are
  linear. The variables themselves might be linear or polynomials. But
  the coefficients and error terms must be linear.
\item
  The error term (\(\epsilon\)) has population mean zero.

  In particular, the error term is normally distributed with the
  population mean of zero. If the population mean of the error term is
  other than zero, it means at least some part of the error term is
  predictable. Such a predictable part should be kept in the regression
  equation itself, not in error term. The error term should contain only
  random error, which can be attributed to chance. When the population
  mean is not equal to zero, statisticians call it ``bias''.
\item
  None of the independent variables is correlated with the error term.

  If any of the independent variable is correlated with the error term,
  it means it is possible to predict the error term. So, in such case,
  it is better to put this information in the regression equation
  itself. As we said in the previous point, the error term should
  contain only the random error, which explains the variability of the
  dependent variable. If the error term also explains the variability of
  any of the independent variable, it is not an error per se.
\item
  Observations in the error term are independent of one another (i.e.,
  not correlated with one another). If error terms are correlated, it
  means it is possible to predict the error term of the next
  observation, using the information of the error term in the first
  observation. And any information, which can be used to predict the
  error term, should be put in the regression equation itself. The error
  terms should contain only the random component.
\item
  The variance in the error terms remain constant for all the
  observations. This assumption is also called homoskedasticity. PENDING
\item
  There is no perfect correlation between any of the independent
  variable with the dependent variable. If there is a perfect
  correlation between any two variables, then one of the two variables
  is unnecessary.
\end{enumerate}

\href{https://www.statology.org/multiple-linear-regression-assumptions/}{The
Five Assumptions of Multiple Linear Regression - Statology}

Model Building Strategies

also see pdf on steps to follow in multiple regression model building

\href{https://bookdown.org/ripberjt/qrmbook/multiple-regression-and-model-building.html}{13
Multiple Regression and Model Building \textbar{} Quantitative Research
Methods for Political Science, Public Policy and Public Administration:
4th Edition With Applications in R (bookdown.org)}

Model Diagnostics

\href{https://faculty.nps.edu/rbassett/_book/multiple-regression-models.html}{Chapter
13 Multiple Regression Models \textbar{} Introduction to Statistics and
Data Science (nps.edu)}

\href{https://web.stanford.edu/class/stats191/notebooks/Diagnostics_for_multiple_regression.html}{Diagnostics\_for\_multiple\_regression
(stanford.edu)}

Interpretation of Results of Multiple Linear Regression

Multicollinearity and Variable Selection

Model Selection Strategies

\bookmarksetup{startatroot}

\hypertarget{regression-analysis-with-dummy-variables}{%
\chapter{Regression Analysis with Dummy
Variables}\label{regression-analysis-with-dummy-variables}}

\begin{longtable}[]{@{}
  >{\centering\arraybackslash}p{(\columnwidth - 0\tabcolsep) * \real{1.0000}}@{}}
\toprule\noalign{}
\begin{minipage}[b]{\linewidth}\centering
Status
\end{minipage} \\
\midrule\noalign{}
\endhead
\bottomrule\noalign{}
\endlastfoot
This chapter is currently a dumping ground for ideas, and we don't
recommend reading it. \\
\end{longtable}

\begin{objectives}{In this chapter, you will learn to}
\begin{itemize}

\item{Incorporating categorical variables in regression analysis}

\item{Creating and interpreting dummy variables in R}

\item{Dummy variable pitfalls and remedies in econometric modeling}

\item{Examples and case studies of dummy variable regression in economics using R}

\end{itemize}

\end{objectives}

\begin{itemize}
\item
  Incorporating categorical variables in regression analysis
\item
  Creating and interpreting dummy variables in R
\item
  Dummy variable pitfalls and remedies in econometric modeling
\item
  Examples and case studies of dummy variable regression in economics
  using R
\end{itemize}

\bookmarksetup{startatroot}

\hypertarget{heteroscedasticity-and-robust-regression}{%
\chapter{Heteroscedasticity and Robust
Regression}\label{heteroscedasticity-and-robust-regression}}

\begin{longtable}[]{@{}
  >{\centering\arraybackslash}p{(\columnwidth - 0\tabcolsep) * \real{1.0000}}@{}}
\toprule\noalign{}
\begin{minipage}[b]{\linewidth}\centering
Status
\end{minipage} \\
\midrule\noalign{}
\endhead
\bottomrule\noalign{}
\endlastfoot
This chapter is currently a dumping ground for ideas, and we don't
recommend reading it. \\
\end{longtable}

\begin{objectives}{In this chapter, you will learn to}
\begin{itemize}

\item{Understanding heteroscedasticity and its implications}

\item{Addressing heteroscedasticity using robust regression techniques in R}

\item{Interpreting robust regression results in an economic context}

\item{Practical examples and exercises showcasing robust regression in econometrics}

\end{itemize}

\end{objectives}

\begin{itemize}
\item
  Understanding heteroscedasticity and its implications
\item
  Addressing heteroscedasticity using robust regression techniques in R
\item
  Interpreting robust regression results in an economic context
\item
  Practical examples and exercises showcasing robust regression in
  econometrics
\end{itemize}

\bookmarksetup{startatroot}

\hypertarget{time-series-regression}{%
\chapter{Time Series Regression}\label{time-series-regression}}

\begin{longtable}[]{@{}
  >{\centering\arraybackslash}p{(\columnwidth - 0\tabcolsep) * \real{1.0000}}@{}}
\toprule\noalign{}
\begin{minipage}[b]{\linewidth}\centering
Status
\end{minipage} \\
\midrule\noalign{}
\endhead
\bottomrule\noalign{}
\endlastfoot
This chapter is currently a dumping ground for ideas, and we don't
recommend reading it. \\
\end{longtable}

\begin{objectives}{In this chapter, you will learn to}
\begin{itemize}

\item{Introduction to time series data in econometrics}

\item{Time series regression models in R for economic analysis}

\item{Dealing with autocorrelation and lagged variables}

\item{Forecasting with time series regression models in R}

\item{Applications of time series regression in economic forecasting}

\end{itemize}

\end{objectives}

\begin{itemize}
\item
  Introduction to time series data in econometrics
\item
  Time series regression models in R for economic analysis
\item
  Dealing with autocorrelation and lagged variables
\item
  Forecasting with time series regression models in R
\item
  Applications of time series regression in economic forecasting
\end{itemize}

\bookmarksetup{startatroot}

\hypertarget{introduction-to-logistic-regression}{%
\chapter{Introduction to Logistic
Regression}\label{introduction-to-logistic-regression}}

\begin{longtable}[]{@{}
  >{\centering\arraybackslash}p{(\columnwidth - 0\tabcolsep) * \real{1.0000}}@{}}
\toprule\noalign{}
\begin{minipage}[b]{\linewidth}\centering
Status
\end{minipage} \\
\midrule\noalign{}
\endhead
\bottomrule\noalign{}
\endlastfoot
This chapter is currently a dumping ground for ideas, and we don't
recommend reading it. \\
\end{longtable}

\begin{objectives}{In this chapter, you will learn to}
\begin{itemize}

\item{Basics of logistic regression in econometrics}

\item{Estimating logistic regression models in R}

\item{Interpreting logistic regression coefficients and odds ratios}

\item{Applications of logistic regression in economic research using R}

\end{itemize}

\end{objectives}

\begin{itemize}
\item
  Basics of logistic regression in econometrics
\item
  Estimating logistic regression models in R
\item
  Interpreting logistic regression coefficients and odds ratios
\item
  Applications of logistic regression in economic research using R
\end{itemize}

\bookmarksetup{startatroot}

\hypertarget{model-evaluation-and-selection}{%
\chapter{Model Evaluation and
Selection}\label{model-evaluation-and-selection}}

\begin{longtable}[]{@{}
  >{\centering\arraybackslash}p{(\columnwidth - 0\tabcolsep) * \real{1.0000}}@{}}
\toprule\noalign{}
\begin{minipage}[b]{\linewidth}\centering
Status
\end{minipage} \\
\midrule\noalign{}
\endhead
\bottomrule\noalign{}
\endlastfoot
This chapter is currently a dumping ground for ideas, and we don't
recommend reading it. \\
\end{longtable}

\begin{objectives}{In this chapter, you will learn to}
\begin{itemize}

\item{Evaluating model performance and goodness-of-fit measures in econometrics}

\item{Validation techniques for econometric regression models}

\item{Comparing and selecting models using information criteria}

\item{Cross-validation and bootstrapping for robust model assessment in econometrics}

\end{itemize}

\end{objectives}

\begin{itemize}
\item
  Evaluating model performance and goodness-of-fit measures in
  econometrics
\item
  Validation techniques for econometric regression models
\item
  Comparing and selecting models using information criteria
\item
  Cross-validation and bootstrapping for robust model assessment in
  econometrics
\end{itemize}

\bookmarksetup{startatroot}

\hypertarget{practical-tips-and-resources-for-econometric-regression}{%
\chapter{Practical Tips and Resources for Econometric
Regression}\label{practical-tips-and-resources-for-econometric-regression}}

\begin{longtable}[]{@{}
  >{\centering\arraybackslash}p{(\columnwidth - 0\tabcolsep) * \real{1.0000}}@{}}
\toprule\noalign{}
\begin{minipage}[b]{\linewidth}\centering
Status
\end{minipage} \\
\midrule\noalign{}
\endhead
\bottomrule\noalign{}
\endlastfoot
This chapter is currently a dumping ground for ideas, and we don't
recommend reading it. \\
\end{longtable}

\begin{objectives}{In this chapter, you will learn to}
\begin{itemize}

\item{Data preparation and preprocessing tips for econometric analysis}

\item{Handling missing data and outliers in regression analysis}

\item{Dealing with endogeneity and instrumental variables}

\item{Additional resources for further learning and practice in econometrics with R}

\end{itemize}

\end{objectives}

\begin{itemize}
\item
  Data preparation and preprocessing tips for econometric analysis
\item
  Handling missing data and outliers in regression analysis
\item
  Dealing with endogeneity and instrumental variables
\item
  Additional resources for further learning and practice in econometrics
  with R
\end{itemize}

\bookmarksetup{startatroot}

\hypertarget{conclusion}{%
\chapter{Conclusion}\label{conclusion}}

\begin{longtable}[]{@{}
  >{\centering\arraybackslash}p{(\columnwidth - 0\tabcolsep) * \real{1.0000}}@{}}
\toprule\noalign{}
\begin{minipage}[b]{\linewidth}\centering
Status
\end{minipage} \\
\midrule\noalign{}
\endhead
\bottomrule\noalign{}
\endlastfoot
This chapter is currently a dumping ground for ideas, and we don't
recommend reading it. \\
\end{longtable}

\begin{objectives}{In this chapter, you will learn to}
\begin{itemize}

\item{Summary of the key concepts covered in the book}

\item{Importance of regression analysis in econometrics and economic research}

\item{Encouragement for further exploration and application of econometric regression using R}

\end{itemize}

\end{objectives}

\begin{itemize}
\item
  Summary of the key concepts covered in the book
\item
  Importance of regression analysis in econometrics and economic
  research
\item
  Encouragement for further exploration and application of econometric
  regression using R
\end{itemize}

\cleardoublepage
\phantomsection
\addcontentsline{toc}{part}{Appendices}
\appendix

\hypertarget{r-packages-for-econometric-regression-analysis-and-additional-resources}{%
\chapter{R packages for econometric regression analysis and additional
resources}\label{r-packages-for-econometric-regression-analysis-and-additional-resources}}

\setcounter{figure}{0} 
\renewcommand{\thefigure}{A.\arabic{figure}}
\setcounter{table}{0} 
\renewcommand{\thetable}{A.\arabic{table}}

This book could never be completed without using many packages. The most
notable of the packages include tidyverse, easystats, AER, lmtest, fpp3,
gujarati5sie, etc. These packages can be installed using the following
commands in R.

\begin{Shaded}
\begin{Highlighting}[numbers=left,,]
\NormalTok{RA\_packages }\OtherTok{\textless{}{-}} \FunctionTok{c}\NormalTok{(}\StringTok{"AER"}\NormalTok{,}
                 \StringTok{"easystats"}\NormalTok{,}
                 \StringTok{"fpp3"}\NormalTok{,}
                 \StringTok{"lmtest"}\NormalTok{,}
                 \StringTok{"ggthemes"}\NormalTok{,}
                 \StringTok{"gt"}\NormalTok{,}
                 \StringTok{"gtsummary"}\NormalTok{,}
                 \StringTok{"patchwork"}\NormalTok{,}
                 \StringTok{"here"}\NormalTok{,}\StringTok{"fs"}\NormalTok{,}
                 \StringTok{"knitr"}\NormalTok{,}
                 \StringTok{"kableExtra"}\NormalTok{)}
\FunctionTok{install.packages}\NormalTok{(RA\_packages)}
\end{Highlighting}
\end{Shaded}

The commands above install the packages into your R system. However, the
functionality of these packages are not added into your R session just
because you installed these packages. You have to load the required
packages specifically whenever you need them.

Imagine you are building a home. You have completed electrification in
your home. But just because you have a fan or an AC in your home doesn't
mean they start automatically. You have to switch on the appliance
whenever you need it. Similarly, you will have to load the R packages
into your session whenever you need those packages. You can load the
package using the \texttt{library(package\_name)} command. For example,
the command to load \texttt{easystats} set of packages would be

\begin{Shaded}
\begin{Highlighting}[numbers=left,,]
\FunctionTok{library}\NormalTok{(easystats)}
\end{Highlighting}
\end{Shaded}

When you run the \texttt{install.packages(package\_name)} command, R
installs the package from Comprehensive R Archive Network (CRAN), which
is the highest authority to decide about R. However, there are many
packages, which are not available on CRAN. These packages can be
downloaded from code sharing platform Github.

To install packages from Github, you need to install one of the three
packages first: \texttt{remotes}, \texttt{devtools} or \texttt{pak}.
After that, you can easily install packages from Github also. Apart from
such packages, the development versions of regular packages (which are
available on CRAN) can also be downloaded from Github. For example, if
you want to install the package ``gujarati5sie'', which is a package
containing data from the book ``Basic Econometrics'' written by Damodar
Gujarati and others, then you can do it as follows:

\begin{Shaded}
\begin{Highlighting}[numbers=left,,]
\CommentTok{\#Install devtools, remotes or pak}
\FunctionTok{install.packages}\NormalTok{(}\StringTok{"remotes"}\NormalTok{)}\CommentTok{\#or install.packages("devltools")}
\CommentTok{\#or install.packages("pak")}
\CommentTok{\#Load the package}
\FunctionTok{library}\NormalTok{(remotes)}\CommentTok{\#or library(devtools) }
\CommentTok{\#or library(pak)}
\CommentTok{\#download the gujarati5sie package }
\CommentTok{\#using one of the above packages}
\NormalTok{remotes}\SpecialCharTok{::}\FunctionTok{install\_github}\NormalTok{(}\StringTok{"bhattmaulik/Gujarati5sie"}\NormalTok{)}
\CommentTok{\#or devtools::install\_github("bhattmaulik/Gujarati5sie") }
\CommentTok{\#or pak::pak("bhattmaulik/Gujarati5sie")}
\end{Highlighting}
\end{Shaded}

\hypertarget{data-sets-used-in-the-books-examples}{%
\chapter{Data sets used in the book's
examples}\label{data-sets-used-in-the-books-examples}}

\setcounter{figure}{0}
\renewcommand{\thefigure}{B.\arabic{figure}} 
\setcounter{table}{0}
\renewcommand{\thetable}{B.\arabic{table}}

\hypertarget{introduction-2}{%
\subsection{Introduction}\label{introduction-2}}

Dummy text

\hypertarget{how-to-build-this-book-locally}{%
\chapter{How to Build This Book
Locally}\label{how-to-build-this-book-locally}}

\setcounter{figure}{0}
\renewcommand{\thefigure}{C.\arabic{figure}}
\setcounter{table}{0}
\renewcommand{\thetable}{C.\arabic{table}}

\hypertarget{introduction-3}{%
\subsection{Introduction}\label{introduction-3}}

If you want to build this book locally on your computer, please download
the entire code from Github. The first step is to visit the website
\texttt{www.github.com/bhattmaulik/RegressionAnalysis}. Here, you will
find the option to download the entire book in a zip file, as shown
below.

\begin{Shaded}
\begin{Highlighting}[numbers=left,,]
\NormalTok{knitr}\SpecialCharTok{::}\FunctionTok{include\_graphics}\NormalTok{(here}\SpecialCharTok{::}\FunctionTok{here}\NormalTok{(}\StringTok{"images"}\NormalTok{,}\StringTok{"download{-}book.png"}\NormalTok{))}
\end{Highlighting}
\end{Shaded}

\begin{figure}[H]

{\centering \includegraphics[width=4.43in,height=\textheight]{images/download-book.png}

}

\caption{\label{fig-download-book}Download the book from Github}

\end{figure}

After you download the zip file, unzip it. This will create a folder in
your computer. Within this folder, double click on the
``RegressionAnalysis.Rproj'' file. This will open the whole project in
RStudio.

After opening this project, go to ``Build'' pane in RStudio. This pane
is generally on the top right of RStudio along with Environment,
History, Connections, Git and Tutorial. In the build pane, click on
``\texttt{Render\ Book}'' and select ``\texttt{HTML\ format}''.

You can also choose to build the book in PDF format, but for that you
will need additional software called \texttt{tinytex}. In order to build
a book in PDF format, R also needs \texttt{LaTeX} compiler. There are
two popular \texttt{LaTeX} compilers: TeX Live and MikTeX. However, they
have their own set of problems for R users. The \texttt{LaTeX} compiler
\texttt{tinytex} attempts to solve many of them.

If you want to build the PDF book, you can install \texttt{tinytex}
through terminal in RStudio using the command
\texttt{quarto\ install\ tinytex}.

\begin{Shaded}
\begin{Highlighting}[numbers=left,,]
\NormalTok{knitr}\SpecialCharTok{::}\FunctionTok{include\_graphics}\NormalTok{(here}\SpecialCharTok{::}\FunctionTok{here}\NormalTok{(}\StringTok{"images"}\NormalTok{,}\StringTok{"build{-}book.png"}\NormalTok{))}
\end{Highlighting}
\end{Shaded}

\begin{figure}[H]

{\centering \includegraphics[width=2.73in,height=\textheight]{images/build-book.png}

}

\caption{\label{fig-build-book}Build book using tinytex}

\end{figure}

If you can't install it this way, or if you want to install it through
traditional way using R code, you can use the following code:

\begin{Shaded}
\begin{Highlighting}[numbers=left,,]
\FunctionTok{install.packages}\NormalTok{(}\StringTok{"tinytex"}\NormalTok{)}
\FunctionTok{library}\NormalTok{(tinytex)}
\NormalTok{tinytex}\SpecialCharTok{::}\FunctionTok{install\_tinytex}\NormalTok{()}
\CommentTok{\#if you want to install all the LaTeX packages, }
\CommentTok{\#you can modify the command to }
\CommentTok{\#tinytex::install\_tinytex(bundle = "TinyTeX{-}2")}
\end{Highlighting}
\end{Shaded}

This confuses some readers because there are two \texttt{tinytex} in
this code: the first tinytex is the \texttt{tinytex} R package. The
other \texttt{tinytex} is the \texttt{LaTeX} compiler \texttt{tinytex}.
So, when we write \texttt{library(tinytex)}, we are calling the R
package tinytex. And when we use the command \texttt{install\_tinytex},
we are installing the \texttt{LaTeX} compiler \texttt{tinytex} using the
R package \texttt{tinytex}. The benefit of this approach is that you get
to select which bundle you want to install. By default, you get to
install the bundle \texttt{TinyTeX-1}, which contains only the most
necessary LaTeX packages. But if you choose the bundle
\texttt{TinyTeX-2}, you can download all the \texttt{LaTeX} packages.

\hypertarget{references}{%
\chapter*{References}\label{references}}
\addcontentsline{toc}{chapter}{References}

\markboth{References}{References}

\thispagestyle{plain}

\hypertarget{refs}{}
\begin{CSLReferences}{1}{0}
\leavevmode\vadjust pre{\hypertarget{ref-ggthemes}{}}%
Arnold, Jeffrey B. 2021. \emph{{ggthemes}: Extra Themes, Scales and
Geoms for {``{ggplot2}''}}.
\url{https://CRAN.R-project.org/package=ggthemes}.

\leavevmode\vadjust pre{\hypertarget{ref-fs}{}}%
Hester, Jim, Hadley Wickham, and Gábor Csárdi. 2023. \emph{{fs}:
Cross-Platform File System Operations Based on {``{libuv}''}}.
\url{https://CRAN.R-project.org/package=fs}.

\leavevmode\vadjust pre{\hypertarget{ref-fpp3}{}}%
Hyndman, Rob. 2023. \emph{Fpp3: Data for {``{Forecasting: Principles and
Practice}''} (3rd Edition)}.
\url{https://CRAN.R-project.org/package=fpp3}.

\leavevmode\vadjust pre{\hypertarget{ref-gt}{}}%
Iannone, Richard, Joe Cheng, Barret Schloerke, Ellis Hughes, Alexandra
Lauer, and JooYoung Seo. 2023. \emph{{gt}: Easily Create
Presentation-Ready Display Tables}.
\url{https://CRAN.R-project.org/package=gt}.

\leavevmode\vadjust pre{\hypertarget{ref-AER}{}}%
Kleiber, Christian, and Achim Zeileis. 2008. \emph{Applied Econometrics
with {R}}. New York: Springer-Verlag.
\url{https://CRAN.R-project.org/package=AER}.

\leavevmode\vadjust pre{\hypertarget{ref-easystats}{}}%
Lüdecke, Daniel, Mattan S. Ben-Shachar, Indrajeet Patil, Brenton M.
Wiernik, Etienne Bacher, Rémi Thériault, and Dominique Makowski. 2022.
{``{easystats}: Framework for Easy Statistical Modeling, Visualization,
and Reporting.''} \emph{CRAN}.
\url{https://easystats.github.io/easystats/}.

\leavevmode\vadjust pre{\hypertarget{ref-here}{}}%
Müller, Kirill. 2020. \emph{{here}: A Simpler Way to Find Your Files}.
\url{https://CRAN.R-project.org/package=here}.

\leavevmode\vadjust pre{\hypertarget{ref-patchwork}{}}%
Pedersen, Thomas Lin. 2022. \emph{{patchwork}: The Composer of Plots}.
\url{https://CRAN.R-project.org/package=patchwork}.

\leavevmode\vadjust pre{\hypertarget{ref-gtsummary}{}}%
Sjoberg, Daniel D., Karissa Whiting, Michael Curry, Jessica A. Lavery,
and Joseph Larmarange. 2021. {``Reproducible Summary Tables with the
Gtsummary Package.''} \emph{{The R Journal}} 13: 570--80.
\url{https://doi.org/10.32614/RJ-2021-053}.

\leavevmode\vadjust pre{\hypertarget{ref-knitr2014}{}}%
Xie, Yihui. 2014. {``{knitr}: A Comprehensive Tool for Reproducible
Research in {R}.''} In \emph{Implementing Reproducible Computational
Research}, edited by Victoria Stodden, Friedrich Leisch, and Roger D.
Peng. Chapman; Hall/CRC.

\leavevmode\vadjust pre{\hypertarget{ref-knitr2015}{}}%
---------. 2015. \emph{Dynamic Documents with {R} and Knitr}. 2nd ed.
Boca Raton, Florida: Chapman; Hall/CRC. \url{https://yihui.org/knitr/}.

\leavevmode\vadjust pre{\hypertarget{ref-knitr2023}{}}%
---------. 2023. \emph{{knitr}: A General-Purpose Package for Dynamic
Report Generation in r}. \url{https://yihui.org/knitr/}.

\leavevmode\vadjust pre{\hypertarget{ref-lmtest}{}}%
Zeileis, Achim, and Torsten Hothorn. 2002. {``Diagnostic Checking in
Regression Relationships.''} \emph{R News} 2 (3): 7--10.
\url{https://CRAN.R-project.org/doc/Rnews/}.

\leavevmode\vadjust pre{\hypertarget{ref-kableExtra}{}}%
Zhu, Hao. 2021. \emph{{kableExtra}: Construct Complex Table with
{``{kable}''} and Pipe Syntax}.
\url{https://CRAN.R-project.org/package=kableExtra}.

\end{CSLReferences}


\backmatter

\end{document}
