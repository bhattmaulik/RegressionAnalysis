% Options for packages loaded elsewhere
\PassOptionsToPackage{unicode,linktoc=all,pdfwindowui,pdfpagemode=FullScreen}{hyperref}
\PassOptionsToPackage{hyphens}{url}
\PassOptionsToPackage{dvipsnames,svgnames,x11names}{xcolor}
%
\documentclass[
  letterpaper,
  twoside,
  openany]{scrbook}

\usepackage{amsmath,amssymb}
\usepackage{iftex}
\ifPDFTeX
  \usepackage[T1]{fontenc}
  \usepackage[utf8]{inputenc}
  \usepackage{textcomp} % provide euro and other symbols
\else % if luatex or xetex
  \usepackage{unicode-math}
  \defaultfontfeatures{Scale=MatchLowercase}
  \defaultfontfeatures[\rmfamily]{Ligatures=TeX,Scale=1}
\fi
\usepackage{lmodern}
\ifPDFTeX\else  
    % xetex/luatex font selection
\fi
% Use upquote if available, for straight quotes in verbatim environments
\IfFileExists{upquote.sty}{\usepackage{upquote}}{}
\IfFileExists{microtype.sty}{% use microtype if available
  \usepackage[]{microtype}
  \UseMicrotypeSet[protrusion]{basicmath} % disable protrusion for tt fonts
}{}
\makeatletter
\@ifundefined{KOMAClassName}{% if non-KOMA class
  \IfFileExists{parskip.sty}{%
    \usepackage{parskip}
  }{% else
    \setlength{\parindent}{0pt}
    \setlength{\parskip}{6pt plus 2pt minus 1pt}}
}{% if KOMA class
  \KOMAoptions{parskip=half}}
\makeatother
\usepackage{xcolor}
\usepackage[lmargin=1in,rmargin=1in,tmargin=1in,bmargin=1in]{geometry}
\setlength{\emergencystretch}{3em} % prevent overfull lines
\setcounter{secnumdepth}{5}
% Make \paragraph and \subparagraph free-standing
\ifx\paragraph\undefined\else
  \let\oldparagraph\paragraph
  \renewcommand{\paragraph}[1]{\oldparagraph{#1}\mbox{}}
\fi
\ifx\subparagraph\undefined\else
  \let\oldsubparagraph\subparagraph
  \renewcommand{\subparagraph}[1]{\oldsubparagraph{#1}\mbox{}}
\fi

\usepackage{color}
\usepackage{fancyvrb}
\newcommand{\VerbBar}{|}
\newcommand{\VERB}{\Verb[commandchars=\\\{\}]}
\DefineVerbatimEnvironment{Highlighting}{Verbatim}{commandchars=\\\{\}}
% Add ',fontsize=\small' for more characters per line
\usepackage{framed}
\definecolor{shadecolor}{RGB}{241,243,245}
\newenvironment{Shaded}{\begin{snugshade}}{\end{snugshade}}
\newcommand{\AlertTok}[1]{\textcolor[rgb]{0.68,0.00,0.00}{#1}}
\newcommand{\AnnotationTok}[1]{\textcolor[rgb]{0.37,0.37,0.37}{#1}}
\newcommand{\AttributeTok}[1]{\textcolor[rgb]{0.40,0.45,0.13}{#1}}
\newcommand{\BaseNTok}[1]{\textcolor[rgb]{0.68,0.00,0.00}{#1}}
\newcommand{\BuiltInTok}[1]{\textcolor[rgb]{0.00,0.23,0.31}{#1}}
\newcommand{\CharTok}[1]{\textcolor[rgb]{0.13,0.47,0.30}{#1}}
\newcommand{\CommentTok}[1]{\textcolor[rgb]{0.37,0.37,0.37}{#1}}
\newcommand{\CommentVarTok}[1]{\textcolor[rgb]{0.37,0.37,0.37}{\textit{#1}}}
\newcommand{\ConstantTok}[1]{\textcolor[rgb]{0.56,0.35,0.01}{#1}}
\newcommand{\ControlFlowTok}[1]{\textcolor[rgb]{0.00,0.23,0.31}{#1}}
\newcommand{\DataTypeTok}[1]{\textcolor[rgb]{0.68,0.00,0.00}{#1}}
\newcommand{\DecValTok}[1]{\textcolor[rgb]{0.68,0.00,0.00}{#1}}
\newcommand{\DocumentationTok}[1]{\textcolor[rgb]{0.37,0.37,0.37}{\textit{#1}}}
\newcommand{\ErrorTok}[1]{\textcolor[rgb]{0.68,0.00,0.00}{#1}}
\newcommand{\ExtensionTok}[1]{\textcolor[rgb]{0.00,0.23,0.31}{#1}}
\newcommand{\FloatTok}[1]{\textcolor[rgb]{0.68,0.00,0.00}{#1}}
\newcommand{\FunctionTok}[1]{\textcolor[rgb]{0.28,0.35,0.67}{#1}}
\newcommand{\ImportTok}[1]{\textcolor[rgb]{0.00,0.46,0.62}{#1}}
\newcommand{\InformationTok}[1]{\textcolor[rgb]{0.37,0.37,0.37}{#1}}
\newcommand{\KeywordTok}[1]{\textcolor[rgb]{0.00,0.23,0.31}{#1}}
\newcommand{\NormalTok}[1]{\textcolor[rgb]{0.00,0.23,0.31}{#1}}
\newcommand{\OperatorTok}[1]{\textcolor[rgb]{0.37,0.37,0.37}{#1}}
\newcommand{\OtherTok}[1]{\textcolor[rgb]{0.00,0.23,0.31}{#1}}
\newcommand{\PreprocessorTok}[1]{\textcolor[rgb]{0.68,0.00,0.00}{#1}}
\newcommand{\RegionMarkerTok}[1]{\textcolor[rgb]{0.00,0.23,0.31}{#1}}
\newcommand{\SpecialCharTok}[1]{\textcolor[rgb]{0.37,0.37,0.37}{#1}}
\newcommand{\SpecialStringTok}[1]{\textcolor[rgb]{0.13,0.47,0.30}{#1}}
\newcommand{\StringTok}[1]{\textcolor[rgb]{0.13,0.47,0.30}{#1}}
\newcommand{\VariableTok}[1]{\textcolor[rgb]{0.07,0.07,0.07}{#1}}
\newcommand{\VerbatimStringTok}[1]{\textcolor[rgb]{0.13,0.47,0.30}{#1}}
\newcommand{\WarningTok}[1]{\textcolor[rgb]{0.37,0.37,0.37}{\textit{#1}}}

\providecommand{\tightlist}{%
  \setlength{\itemsep}{0pt}\setlength{\parskip}{0pt}}\usepackage{longtable,booktabs,array}
\usepackage{calc} % for calculating minipage widths
% Correct order of tables after \paragraph or \subparagraph
\usepackage{etoolbox}
\makeatletter
\patchcmd\longtable{\par}{\if@noskipsec\mbox{}\fi\par}{}{}
\makeatother
% Allow footnotes in longtable head/foot
\IfFileExists{footnotehyper.sty}{\usepackage{footnotehyper}}{\usepackage{footnote}}
\makesavenoteenv{longtable}
\usepackage{graphicx}
\makeatletter
\def\maxwidth{\ifdim\Gin@nat@width>\linewidth\linewidth\else\Gin@nat@width\fi}
\def\maxheight{\ifdim\Gin@nat@height>\textheight\textheight\else\Gin@nat@height\fi}
\makeatother
% Scale images if necessary, so that they will not overflow the page
% margins by default, and it is still possible to overwrite the defaults
% using explicit options in \includegraphics[width, height, ...]{}
\setkeys{Gin}{width=\maxwidth,height=\maxheight,keepaspectratio}
% Set default figure placement to htbp
\makeatletter
\def\fps@figure{htbp}
\makeatother

\providecommand{\abstractname}{Learning Objectives} % not in scrbook class
\newenvironment{objectives}[1]{%
	\hrule
	\vspace{5pt}
	\small\textbf{\abstractname: } 
	\newline
	\vspace{0.1cm}
	%\small\emph #1     %  emph takes an argument
	\small\emph{#1} % or \small\textit{#1} 
	\itshape % use this if you want the text to be in italics
}{%
	\vspace{5pt}
	\hrule
	\vspace{0.6cm}
}
\makeatletter
\makeatother
\makeatletter
\@ifpackageloaded{bookmark}{}{\usepackage{bookmark}}
\makeatother
\makeatletter
\@ifpackageloaded{caption}{}{\usepackage{caption}}
\AtBeginDocument{%
\ifdefined\contentsname
  \renewcommand*\contentsname{Table of contents}
\else
  \newcommand\contentsname{Table of contents}
\fi
\ifdefined\listfigurename
  \renewcommand*\listfigurename{List of Figures}
\else
  \newcommand\listfigurename{List of Figures}
\fi
\ifdefined\listtablename
  \renewcommand*\listtablename{List of Tables}
\else
  \newcommand\listtablename{List of Tables}
\fi
\ifdefined\figurename
  \renewcommand*\figurename{Figure}
\else
  \newcommand\figurename{Figure}
\fi
\ifdefined\tablename
  \renewcommand*\tablename{Table}
\else
  \newcommand\tablename{Table}
\fi
}
\@ifpackageloaded{float}{}{\usepackage{float}}
\floatstyle{ruled}
\@ifundefined{c@chapter}{\newfloat{codelisting}{h}{lop}}{\newfloat{codelisting}{h}{lop}[chapter]}
\floatname{codelisting}{Listing}
\newcommand*\listoflistings{\listof{codelisting}{List of Listings}}
\makeatother
\makeatletter
\@ifpackageloaded{caption}{}{\usepackage{caption}}
\@ifpackageloaded{subcaption}{}{\usepackage{subcaption}}
\makeatother
\makeatletter
\@ifpackageloaded{tcolorbox}{}{\usepackage[skins,breakable]{tcolorbox}}
\makeatother
\makeatletter
\@ifundefined{shadecolor}{\definecolor{shadecolor}{rgb}{.97, .97, .97}}
\makeatother
\makeatletter
\makeatother
\makeatletter
\makeatother
\ifLuaTeX
  \usepackage{selnolig}  % disable illegal ligatures
\fi
\IfFileExists{bookmark.sty}{\usepackage{bookmark}}{\usepackage{hyperref}}
\IfFileExists{xurl.sty}{\usepackage{xurl}}{} % add URL line breaks if available
\urlstyle{same} % disable monospaced font for URLs
\hypersetup{
  pdftitle={Regression Analysis With R and Easystats},
  pdfauthor={Maulik Bhatt},
  colorlinks=true,
  linkcolor={Maroon},
  filecolor={Maroon},
  citecolor={Blue},
  urlcolor={Blue},
  pdfcreator={LaTeX via pandoc}}

\title{Regression Analysis With R and Easystats}
\author{Maulik Bhatt}
\date{17/06/2023}

\begin{document}
\frontmatter
\maketitle
\ifdefined\Shaded\renewenvironment{Shaded}{\begin{tcolorbox}[sharp corners, frame hidden, breakable, enhanced, interior hidden, boxrule=0pt, borderline west={3pt}{0pt}{shadecolor}]}{\end{tcolorbox}}\fi

\renewcommand*\contentsname{Contents}
{
\hypersetup{linkcolor=}
\setcounter{tocdepth}{2}
\tableofcontents
}
\listoffigures
\listoftables
\mainmatter
\bookmarksetup{startatroot}

\hypertarget{preface}{%
\chapter*{Preface}\label{preface}}
\addcontentsline{toc}{chapter}{Preface}

\markboth{Preface}{Preface}

\bookmarksetup{startatroot}

\hypertarget{acknowledgement}{%
\chapter*{Acknowledgement}\label{acknowledgement}}
\addcontentsline{toc}{chapter}{Acknowledgement}

\markboth{Acknowledgement}{Acknowledgement}

I would like to express my heartfelt gratitude to the individuals who
have played a significant role in the creation of this book on
Regression Analysis with R. Their guidance, knowledge, and support have
been invaluable throughout this journey.

First and foremost, I extend my deepest appreciation to my esteemed
professors, {[}Professor's Name{]}, {[}Professor's Name{]}, and
{[}Professor's Name{]}. Their expertise in statistics and econometrics
has shaped my understanding of quantitative analysis and provided a
solid foundation for this book. Their dedication to teaching and their
unwavering commitment to fostering academic excellence have been
instrumental in my growth as a student of econometrics.

I would like to extend special thanks to my colleague and dear friend,
{[}Colleague's Name{]}. Their introduction to the world of R programming
language opened up a world of possibilities for me. Their patience,
willingness to share their knowledge, and countless hours spent
assisting me with R-related challenges have been indispensable. I am
grateful for their unwavering support and the collaborative environment
we fostered, which significantly enriched my learning experience.

A heartfelt appreciation goes to the developers of R, an open-source
programming language that has revolutionized the field of data analysis
and statistical computing. Their tireless efforts in creating and
continuously improving R have made it an indispensable tool for
researchers and analysts worldwide. Without their dedication, this book
would not have been possible.

I would also like to express my gratitude to the developers of the R
packages that have been instrumental in the analysis and visualization
techniques presented in this book. Their commitment to excellence,
innovation, and user-friendly implementations has immensely contributed
to the field of regression analysis. Their packages have not only
expanded the capabilities of R but have also facilitated the seamless
integration of econometric methodologies into practical applications.

Finally, I extend my heartfelt thanks to my family and friends who have
supported me throughout this writing process. Their encouragement,
understanding, and belief in my abilities have been a constant source of
motivation.

To all those mentioned above, and to anyone else who has contributed to
this book in any way, I offer my deepest appreciation. Your guidance,
knowledge, and support have been invaluable in shaping this work and
have helped me fulfill my goal of sharing the beauty and importance of
regression analysis with R.

Thank you.

{[}Your Name{]}

\part{BASICS OF ECONOMETRICS AND R}

\hypertarget{introduction}{%
\chapter{Introduction}\label{introduction}}

\begin{objectives}{In this chapter, you will learn to}
\begin{itemize}

\item{Importance of regression analysis in econometrics}

\item{Overview of the book's structure and goals}

\item{Introduction to R programming language and its relevance to econometric analysis}

\end{itemize}

\end{objectives}

This chapter introduces the basics of Econometrics and R to a novice
user. Econometrics is a combination of two words: Econo + Metric. Econo
refers to concepts of economics, while metric refers to measurement.
Let's take an example of the law of demand from microeconomic theory. We
know the demand of a commodity will decrease if the price of the
commodity increases. But we don't know how much the demand will
decrease, given the increase in the price is one unit (i.e., one Dollar,
or one Euro, etc.). In Econometrics, we measure such increase or
decrease using various experiments.

\hypertarget{introduction-to-econometrics-and-regression-analysis}{%
\chapter{Introduction to Econometrics and Regression
Analysis}\label{introduction-to-econometrics-and-regression-analysis}}

\begin{objectives}{In this chapter, you will learn to}
\begin{itemize}

\item{Understanding the basics of econometrics}

\item{Role of regression analysis in econometric modeling}

\item{Overview of the regression analysis process}

\item{Introduction to R for econometric analysis}

\end{itemize}

\end{objectives}

Sample text here

\hypertarget{getting-started-with-r-for-econometrics}{%
\chapter{Getting Started with R for
Econometrics}\label{getting-started-with-r-for-econometrics}}

\begin{objectives}{In this chapter, you will learn to}
\begin{itemize}

\item{Introduction to R programming language and its ecosystem}

\item{Setting up the R environment and installing necessary packages}

\item{Loading and manipulating data in R for econometric analysis}

\item{Exploring data visualization techniques using R}

\end{itemize}

\end{objectives}

\part{ISSUES IN REGRESSION ANALYSIS}

\hypertarget{simple-linear-regression}{%
\chapter{Simple Linear Regression}\label{simple-linear-regression}}

\begin{objectives}{In this chapter, you will learn to}
\begin{itemize}

\item{Understanding the principles of simple linear regression}

\item{Performing simple linear regression in R for econometric analysis}

\item{Interpreting regression results in the context of economic variables}

\item{Assessing model assumptions and addressing violations}

\item{Practical examples and exercises using R}

\end{itemize}

\end{objectives}

\begin{itemize}
\item
  Understanding the principles of simple linear regression
\item
  Performing simple linear regression in R for econometric analysis
\item
  Interpreting regression results in the context of economic variables
\item
  Assessing model assumptions and addressing violations
\item
  Practical examples and exercises using R
\end{itemize}

\hypertarget{multiple-linear-regression}{%
\chapter{Multiple Linear Regression}\label{multiple-linear-regression}}

\begin{objectives}{In this chapter, you will learn to}
\begin{itemize}

\item{Extending regression analysis to multiple independent variables}

\item{Building and interpreting multiple linear regression models in R}

\item{Handling multicollinearity and selecting significant predictors in an economic context}

\item{Model evaluation and diagnostics in econometric regression}

\item{Application of multiple linear regression in economic analysis using R}

\end{itemize}

\end{objectives}

\begin{itemize}
\item
  Extending regression analysis to multiple independent variables
\item
  Building and interpreting multiple linear regression models in R
\item
  Handling multicollinearity and selecting significant predictors in an
  economic context
\item
  Model evaluation and diagnostics in econometric regression
\item
  Application of multiple linear regression in economic analysis using R
\end{itemize}

\hypertarget{regression-analysis-with-dummy-variables}{%
\chapter{Regression Analysis with Dummy
Variables}\label{regression-analysis-with-dummy-variables}}

\begin{objectives}{In this chapter, you will learn to}
\begin{itemize}

\item{Incorporating categorical variables in regression analysis}

\item{Creating and interpreting dummy variables in R}

\item{Dummy variable pitfalls and remedies in econometric modeling}

\item{Examples and case studies of dummy variable regression in economics using R}

\end{itemize}

\end{objectives}

\begin{itemize}
\item
  Incorporating categorical variables in regression analysis
\item
  Creating and interpreting dummy variables in R
\item
  Dummy variable pitfalls and remedies in econometric modeling
\item
  Examples and case studies of dummy variable regression in economics
  using R
\end{itemize}

\hypertarget{heteroscedasticity-and-robust-regression}{%
\chapter{Heteroscedasticity and Robust
Regression}\label{heteroscedasticity-and-robust-regression}}

\begin{objectives}{In this chapter, you will learn to}
\begin{itemize}

\item{Understanding heteroscedasticity and its implications}

\item{Addressing heteroscedasticity using robust regression techniques in R}

\item{Interpreting robust regression results in an economic context}

\item{Practical examples and exercises showcasing robust regression in econometrics}

\end{itemize}

\end{objectives}

\begin{itemize}
\item
  Understanding heteroscedasticity and its implications
\item
  Addressing heteroscedasticity using robust regression techniques in R
\item
  Interpreting robust regression results in an economic context
\item
  Practical examples and exercises showcasing robust regression in
  econometrics
\end{itemize}

\hypertarget{time-series-regression}{%
\chapter{Time Series Regression}\label{time-series-regression}}

\begin{objectives}{In this chapter, you will learn to}
\begin{itemize}

\item{Introduction to time series data in econometrics}

\item{Time series regression models in R for economic analysis}

\item{Dealing with autocorrelation and lagged variables}

\item{Forecasting with time series regression models in R}

\item{Applications of time series regression in economic forecasting}

\end{itemize}

\end{objectives}

\begin{itemize}
\item
  Introduction to time series data in econometrics
\item
  Time series regression models in R for economic analysis
\item
  Dealing with autocorrelation and lagged variables
\item
  Forecasting with time series regression models in R
\item
  Applications of time series regression in economic forecasting
\end{itemize}

\hypertarget{introduction-to-logistic-regression}{%
\chapter{Introduction to Logistic
Regression}\label{introduction-to-logistic-regression}}

\begin{objectives}{In this chapter, you will learn to}
\begin{itemize}

\item{Basics of logistic regression in econometrics}

\item{Estimating logistic regression models in R}

\item{Interpreting logistic regression coefficients and odds ratios}

\item{Applications of logistic regression in economic research using R}

\end{itemize}

\end{objectives}

\begin{itemize}
\item
  Basics of logistic regression in econometrics
\item
  Estimating logistic regression models in R
\item
  Interpreting logistic regression coefficients and odds ratios
\item
  Applications of logistic regression in economic research using R
\end{itemize}

\hypertarget{model-evaluation-and-selection}{%
\chapter{Model Evaluation and
Selection}\label{model-evaluation-and-selection}}

\begin{objectives}{In this chapter, you will learn to}
\begin{itemize}

\item{Evaluating model performance and goodness-of-fit measures in econometrics}

\item{Validation techniques for econometric regression models}

\item{Comparing and selecting models using information criteria}

\item{Cross-validation and bootstrapping for robust model assessment in econometrics}

\end{itemize}

\end{objectives}

\begin{itemize}
\item
  Evaluating model performance and goodness-of-fit measures in
  econometrics
\item
  Validation techniques for econometric regression models
\item
  Comparing and selecting models using information criteria
\item
  Cross-validation and bootstrapping for robust model assessment in
  econometrics
\end{itemize}

\hypertarget{practical-tips-and-resources-for-econometric-regression}{%
\chapter{Practical Tips and Resources for Econometric
Regression}\label{practical-tips-and-resources-for-econometric-regression}}

\begin{objectives}{In this chapter, you will learn to}
\begin{itemize}

\item{Data preparation and preprocessing tips for econometric analysis}

\item{Handling missing data and outliers in regression analysis}

\item{Dealing with endogeneity and instrumental variables}

\item{Additional resources for further learning and practice in econometrics with R}

\end{itemize}

\end{objectives}

\begin{itemize}
\item
  Data preparation and preprocessing tips for econometric analysis
\item
  Handling missing data and outliers in regression analysis
\item
  Dealing with endogeneity and instrumental variables
\item
  Additional resources for further learning and practice in econometrics
  with R
\end{itemize}

\hypertarget{conclusion}{%
\chapter{Conclusion}\label{conclusion}}

\begin{objectives}{In this chapter, you will learn to}
\begin{itemize}

\item{Summary of the key concepts covered in the book}

\item{Importance of regression analysis in econometrics and economic research}

\item{Encouragement for further exploration and application of econometric regression using R}

\end{itemize}

\end{objectives}

\begin{itemize}
\item
  Summary of the key concepts covered in the book
\item
  Importance of regression analysis in econometrics and economic
  research
\item
  Encouragement for further exploration and application of econometric
  regression using R
\end{itemize}

\part{APPENDICES}

\pagenumbering{Roman}

\hypertarget{appendix-a-r-packages-for-econometric-regression-analysis-and-additional-resources}{%
\chapter*{Appendix A: R packages for econometric regression analysis and
additional
resources}\label{appendix-a-r-packages-for-econometric-regression-analysis-and-additional-resources}}
\addcontentsline{toc}{chapter}{Appendix A: R packages for econometric
regression analysis and additional resources}

\markboth{Appendix A: R packages for econometric regression analysis and
additional resources}{Appendix A: R packages for econometric regression
analysis and additional resources}

\hypertarget{appendix-b-data-sets-used-in-the-books-examples}{%
\chapter*{Appendix B: Data sets used in the book's
examples}\label{appendix-b-data-sets-used-in-the-books-examples}}
\addcontentsline{toc}{chapter}{Appendix B: Data sets used in the book's
examples}

\markboth{Appendix B: Data sets used in the book's examples}{Appendix B:
Data sets used in the book's examples}

\hypertarget{appendix-c-r-code-snippets-and-tutorials-for-reference}{%
\chapter*{Appendix C: R code snippets and tutorials for
reference}\label{appendix-c-r-code-snippets-and-tutorials-for-reference}}
\addcontentsline{toc}{chapter}{Appendix C: R code snippets and tutorials
for reference}

\markboth{Appendix C: R code snippets and tutorials for
reference}{Appendix C: R code snippets and tutorials for reference}

The whole book is published on bookdown.org website. You can read the
book online on this website. You can also download the source code of
this book in multiple ways. The easiest way is to visit the website
github, and Alternatively, you can download the entire source code of
the book from github. Please run the following code to download the
book:

\begin{Shaded}
\begin{Highlighting}[numbers=left,,]
\CommentTok{\#install.packages("devtools")}
\CommentTok{\#devtools::install\_github("bhattmaulik/RegressionAnalysis)}
\end{Highlighting}
\end{Shaded}



\backmatter

\end{document}
